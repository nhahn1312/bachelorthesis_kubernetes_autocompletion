\chapter{Anhang}\label{ch:appendix}

\begin{figure}[htp] % htp = hier (h), top (t), oder auf einer eigenen Seite (p).
    \centering
    \includesvg[width=0.8\textwidth]{images/language-service-interface.svg} % width immer angeben!
    \caption{Klassendiagramm des Interface ``LanguageService''}
    \label{fig:language-service-interface-defition}
\end{figure}

\vspace{0.5cm}

\begin{figure}[htp] % htp = hier (h), top (t), oder auf einer eigenen Seite (p).
    \centering
    \includesvg[width=0.35\textwidth]{images/class-diagram-json-document.svg} % width immer angeben!
    \caption{Klassendiagramm der Klasse ``JSONDocument''}
    \label{fig:class-diagram-json-document}
\end{figure}

\begin{figure}[htp] % htp = hier (h), top (t), oder auf einer eigenen Seite (p).
    \centering
    \includesvg[angle=90, width=0.73\paperheight]{images/class-diagram-lsp-server.svg} % width immer angeben!
    \caption{Klassendiagramm \acs{lsp}-Server}
    \label{fig:class-diagram-lsp-Server}
\end{figure}

\begin{figure}[htp] % htp = hier (h), top (t), oder auf einer eigenen Seite (p).
    \centering
    \includesvg[width=0.8\textwidth]{images/flowchart-get-node-from-offset.svg} % width immer angeben!
    \caption{Ablaufdiagramm der Methode ``getNodeFromOffset''}
    \floatfoot{Im Ablaufdiagramm steht ``CP'' für die Position des Cursors und ``K'' kann einen Knoten aus dem \ac{ast} eines
        \ac{yaml}-\textit{Documents} enthalten.}
    \label{fig:flowchart-get-node-from-offset}
\end{figure}





