\chapter{Anhang}\label{ch:appendix}

\begin{listing}[htp]
    \begin{minted}[fontsize=\small]{yaml}
---
apiVersion: v1
kind: Pod
metadata:
  name: pod-exp
  namespace: namespace-exp
  labels:
    app: backend  
status:
  message: experiment
spec: 
  hostPID: true
  hostIPC: true
  hostNetwork: true
  automountServiceAccountToken: false
  containers:
    - name: node-server
      image: nginx:stable-alpine
      env:
        - name: PASSWORD
          value: "super_secret"
        - name: ELASTIC
          value: "http://localhost:9200" 
  volumes:
    - name: dockersocket
      hostPath:
        path: /var/run/docker.sock
    \end{minted}
    \caption{pod.yaml}
    \label{lst:pod-yaml-file}
\end{listing}

\begin{listing}[htp]
    \begin{minted}[fontsize=\small]{yaml}
---
apiVersion: apps/v1
kind: Deployment
metadata:
  namespace: namespace-exp
  name: deployment-exp
  labels:
    app: frontend
spec:
  selector:
    matchLabels:
      app: frontend
  template:
    metadata:
      labels:
        app: frontend
    spec:
      containers:
        - name: frontend-exp
          image: nginx:stable-alpine
          securityContext:
            allowPrivilegeEscalation: false
            privileged: true
            capabilities:
              add:
                - CAP_SYS_ADMIN
                - CAP_SYS_MODULE
      volumes:
        - name: secret
          secret:
            secretName: secret-exp 
    \end{minted}
    \caption{deployment.yaml}
    \label{lst:deployment-yaml-file}
\end{listing}


\begin{figure}[htp] % htp = hier (h), top (t), oder auf einer eigenen Seite (p).
    \centering
    \includesvg[width=0.8\textwidth]{images/language-service-interface.svg} % width immer angeben!
    \caption{Klassendiagramm des Interface ``LanguageService''}
    \label{fig:language-service-interface-defition}
\end{figure}

\vspace{0.5cm}

\begin{figure}[htp] % htp = hier (h), top (t), oder auf einer eigenen Seite (p).
    \centering
    \includesvg[width=0.35\textwidth]{images/class-diagram-json-document.svg} % width immer angeben!
    \caption{Klassendiagramm der Klasse ``JSONDocument''}
    \label{fig:class-diagram-json-document}
\end{figure}

\begin{figure}[htp] % htp = hier (h), top (t), oder auf einer eigenen Seite (p).
    \centering
    \includesvg[angle=90, width=0.73\paperheight]{images/class-diagram-lsp-server.svg} % width immer angeben!
    \caption{Klassendiagramm \acs{lsp}-Server}
    \label{fig:class-diagram-lsp-Server}
\end{figure}

\begin{figure}[htp] % htp = hier (h), top (t), oder auf einer eigenen Seite (p).
    \centering
    \includesvg[width=0.8\textwidth]{images/flowchart-get-node-from-offset.svg} % width immer angeben!
    \caption{Ablaufdiagramm der Methode ``getNodeFromOffset''}
    \floatfoot{Im Ablaufdiagramm steht ``CP'' für die Position des Cursors und ``K'' kann einen Knoten aus dem \ac{ast} eines
        \ac{yaml}-\textit{Documents} enthalten.}
    \label{fig:flowchart-get-node-from-offset}
\end{figure}





