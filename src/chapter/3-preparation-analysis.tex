\chapter{Vorbereitende Analyse}\label{ch:preparation-analysis}

In dem vorliegenden Kapitel werden die Probleme und Herausforderungen beim Erstellen von Kubernetes \textit{Manifest}-Dateien anhand von wissenschaftlichen Arbeiten
und Internetquellen, sogenannter grauer Literatur, aufgezeigt. Anschließend werden bereits vorhandene Lösungen für die zuvor herausgearbeiteten Punkte vorgestellt.
Zuletzt wird evaluiert, ob den vorhandenen Lösungen Funktionen fehlen.

\section{Probleme und Herausforderungen}

Als Erstes wurde nach wissenschaftlichen Arbeiten gesucht, die sich mit Kubernetes Konfigurationen und den Kubernetes \textit{Manifest}-Dateien
beschäftigen. Dazu wurden folgende Suchbegriffe verwendet:
\begin{itemize}
    \setlength\itemsep{-0.5cm}
    \item ``kubernetes configuration''
    \item ``kubernetes manifests''
    \item ``kubernetes challenges''
    \item ``kubernetes deployment''
    \item ``kubernetes problems''
    \item ``kubernetes yaml''
\end{itemize}
Die Suche wurde auf folgenden Seiten durchgeführt:
\begin{itemize}
    \setlength\itemsep{-0.5cm}
    \item ACM Digitial Library (\url{https://dl.acm.org/})
    \item IEEE Xplore (\url{https://ieeexplore.ieee.org/})
    \item Google Scholar (\url{https://scholar.google.de/})
\end{itemize}
Dabei wurden fünf Arbeiten gefunden, die das gesuchte Thema behandeln. Zwei Arbeiten zeigen dabei alternative Wege zur Konfiguration von Kubernetes auf.
Der Inhalt der Arbeiten wurde bereits in Kapitel~\ref{sec:introduction-related-work} beschrieben. Die anderen drei Arbeiten untersuchen Sicherheitsmängel,
die durch die Erstellung von Kubernetes \textit{Manifest}-Dateien auftreten können.
Nur die Arbeit ``Isopod: An Expressive DSL for Kubernetes Configuration'' erwähnt ein Problem, welches keinen Sicherheitsmangel dargestellt, bei der Erstellung von Kubernetes \textit{Manifest}-Dateien
in einem Satz.
\\
Aufgrund der geringen Repräsentation in wissenschaftlichen Arbeiten wurde eine Websuche durchgeführt, um die Probleme und Herausforderungen beim Erstellen
der Kubernetes Konfigurationsdateien zu untersuchen.
Die Suche wurde mit den zuvor verwendeten Suchbegriffen durchgeführt und um folgende Suchbegriffe erweitert:
\begin{itemize}
    \setlength\itemsep{-0.5cm}
    \item ``kubernetes yaml manifest''
    \item ``kubernetes configuration challenges''
    \item ``kubernetes deployment challenges''
    \item ``kubernetes manifest challenges''
    \item ``kubernetes manifest problems''
\end{itemize}
Mit Hilfe der Ergebnisse der Websuche konnte eine Liste der Probleme und Herausforderungen beim Schreiben von Kubernetes \textit{Manifest}-Dateien erstellt werden:
\begin{description}
    \item[Herausforderung 1]{- \ac{yaml}-Syntax\\}
          Es kommt zu Herausforderungen bei Einhalten der \ac{yaml}-Syntax~\cite{10.1145/3357223.3365759,the-chief-io-kubernetes-challenges,techtarget-kubernetes-challenges,monokle-kubernetes-challenges}.
          Die inkorrekte Einrückung kann \ac{zB} ein Problem darstellen~\cite{10.1145/3357223.3365759,monokle-kubernetes-challenges}.
    \item[Herausforderung 2]{- Komplexe Syntax\\}
          Die Syntax der \textit{Manifest}-Dateien stellt eine Herausforderung dar~\cite{dev-to-kubernetes-challenges,kubetools-io-kubernetes-manifest-management,entwickler-de-kubernetes-problems,newstack-io-kubernetes-manifest-lifecycle,10.1145/3357223.3365759,techtarget-kubernetes-challenges,monokle-kubernetes-challenges,kubernetes-config-problems}.
          \textit{Manifest}-Dateien nutzen eine spezielle Syntax, die jedoch inkonsistent und mehrdeutig ist~\cite{dev-to-kubernetes-challenges}.
          Jede Definition eines Kubernetes Objektes hat eine eigene Konfiguration, die der Nutzer kennen muss~\cite{kubetools-io-kubernetes-manifest-management,entwickler-de-kubernetes-problems}.
          Jedes Kubernetes Objekt hat \ac{zB} eine andere Menge von Pflichtschlüsseln, wenn der Nutzer diese nicht kennt, gibt es Probleme beim Ausrollen der Konfiguration~\cite{10.1145/3357223.3365759, techtarget-kubernetes-challenges}.
          Außerdem verursachen vom Nutzer falsch eingegebene oder nicht existierende \textit{Mapping}-Schlüssel Probleme~\cite{monokle-kubernetes-challenges}.
    \item[Herausforderung 3]{- Erkennen von Sicherheitslücken\\}
          Eine weitere Herausforderung ist das Erkennen von Sicherheitslücken beim Schreiben der Kubernetes Konfigurationsdateien sein~\cite{9476056,dynatrace-kubernetes-security-challenges,10.1145/3468264.3473495,10.1145/3579639}.
          Eine Sicherheitslücke kann \ac{zB} ein Passwort im Klartext in einer \textit{Manifest}-Datei sein~\cite{10.1145/3579639}.
    \item[Herausforderung 4]{- Abhängigkeiten zwischen Objekten\\}
          Die Abhängigkeit der Kubernetes Objekte untereinander stellt den Nutzer vor eine Herausforderung~\cite{dev-to-kubernetes-challenges,kubetools-io-kubernetes-manifest-management,spacelift-io-kubernetes-challenges,qovery-kubernetes-challenges}.
          Es kann \ac{zB} zu einer inkorrekten Konfiguration eines \textit{Labels} einer Ressource kommen~\cite{spacelift-io-kubernetes-challenges,qovery-kubernetes-challenges}.
          Dieses Problem kann auftreten, wenn der Nutzer nicht über alle verfügbaren \textit{Labels} informiert ist.
\end{description}

\section{Vorhandene Lösungen}

\textbf{kubeconform}

\textbf{Monokle}

\textbf{Kubernetes Lens}

\textbf{\acs{vscode}-\ac{yaml}}

\textbf{Kubernetes Manifest Editor}

\textbf{KubeLinter}