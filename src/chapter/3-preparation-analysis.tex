\chapter{Vorbereitende Analyse}\label{ch:preparation-analysis}

In dem vorliegenden Kapitel werden die Probleme und Herausforderungen beim Erstellen von Kubernetes \textit{Manifest}-Dateien anhand von wissenschaftlichen Arbeiten
und Internetquellen, sogenannter grauer Literatur, aufgezeigt. Anschließend werden bereits vorhandene Lösungen für die zuvor herausgearbeiteten Punkte vorgestellt.
Zuletzt wird evaluiert, ob den vorhandenen Lösungen Funktionen fehlen.

\section{Probleme und Herausforderungen}

Als Erstes wurde nach wissenschaftlichen Arbeiten gesucht, die sich mit Kubernetes Konfigurationen und den Kubernetes \textit{Manifest}-Dateien
beschäftigen. Dazu wurden folgende Suchbegriffe verwendet:
\begin{itemize}
    \setlength\itemsep{-0.5cm}
    \item ``kubernetes configuration''
    \item ``kubernetes manifests''
    \item ``kubernetes challenges''
    \item ``kubernetes deployment''
    \item ``kubernetes problems''
    \item ``kubernetes yaml''
\end{itemize}
Die Suche wurde auf folgenden Seiten durchgeführt:
\begin{itemize}
    \setlength\itemsep{-0.5cm}
    \item ACM Digitial Library (\url{https://dl.acm.org/})
    \item IEEE Xplore (\url{https://ieeexplore.ieee.org/})
    \item Google Scholar (\url{https://scholar.google.de/})
\end{itemize}
Dabei wurden fünf Arbeiten gefunden, die das gesuchte Thema behandeln. Zwei Arbeiten zeigen dabei alternative Wege zur Konfiguration von Kubernetes auf.
Der Inhalt der Arbeiten wurde bereits in Kapitel~\ref{sec:introduction-related-work} beschrieben. Die anderen drei Arbeiten untersuchen Sicherheitsmängel,
die durch die Erstellung von Kubernetes \textit{Manifest}-Dateien auftreten können.
Nur die Arbeit ``Isopod: An Expressive DSL for Kubernetes Configuration'' erwähnt ein Problem, welches keinen Sicherheitsmangel dargestellt, bei der Erstellung von Kubernetes \textit{Manifest}-Dateien
in einem Satz.
\\
Aufgrund der geringen Repräsentation in wissenschaftlichen Arbeiten wurde eine Websuche durchgeführt, um die Probleme und Herausforderungen beim Erstellen
der Kubernetes Konfigurationsdateien zu untersuchen.
Die Suche wurde mit den zuvor verwendeten Suchbegriffen durchgeführt und um folgende Suchbegriffe erweitert:
\begin{itemize}
    \setlength\itemsep{-0.5cm}
    \item ``kubernetes yaml manifest''
    \item ``kubernetes configuration challenges''
    \item ``kubernetes deployment challenges''
    \item ``kubernetes manifest challenges''
    \item ``kubernetes manifest problems''
\end{itemize}
Mit Hilfe der Ergebnisse der Websuche konnte eine Liste der Probleme und Herausforderungen beim Schreiben von Kubernetes \textit{Manifest}-Dateien erstellt werden:
\begin{description}
    \item[Herausforderung 1]{- \ac{yaml}-Syntax\\}
          Es kommt zu Herausforderungen bei Einhalten der \ac{yaml}-Syntax~\cite{10.1145/3357223.3365759,the-chief-io-kubernetes-challenges,techtarget-kubernetes-challenges,monokle-kubernetes-challenges}.
          Die inkorrekte Einrückung kann \ac{zB} ein Problem darstellen~\cite{10.1145/3357223.3365759,monokle-kubernetes-challenges}.
    \item[Herausforderung 2]{- Komplexe Syntax\\}
          Die Syntax der \textit{Manifest}-Dateien stellt eine Herausforderung dar~\cite{dev-to-kubernetes-challenges,kubetools-io-kubernetes-manifest-management,entwickler-de-kubernetes-problems,newstack-io-kubernetes-manifest-lifecycle,10.1145/3357223.3365759,techtarget-kubernetes-challenges,monokle-kubernetes-challenges,kubernetes-config-problems}.
          \textit{Manifest}-Dateien nutzen eine spezielle Syntax, die jedoch inkonsistent und mehrdeutig ist~\cite{dev-to-kubernetes-challenges}.
          Jede Definition eines Kubernetes Objektes hat eine eigene Konfiguration, die der Nutzer kennen muss~\cite{kubetools-io-kubernetes-manifest-management,entwickler-de-kubernetes-problems}.
          Jedes Kubernetes Objekt hat \ac{zB} eine andere Menge von Pflichtschlüsseln, wenn der Nutzer diese nicht kennt, gibt es Probleme beim Ausrollen der Konfiguration~\cite{10.1145/3357223.3365759, techtarget-kubernetes-challenges}.
          Außerdem verursachen vom Nutzer falsch eingegebene oder nicht existierende \textit{Mapping}-Schlüssel Probleme~\cite{monokle-kubernetes-challenges}.
    \item[Herausforderung 3]{- Erkennen von Sicherheitslücken\\}
          Eine weitere Herausforderung ist das Erkennen von Sicherheitslücken beim Schreiben der Kubernetes Konfigurationsdateien sein~\cite{9476056,dynatrace-kubernetes-security-challenges,10.1145/3468264.3473495,10.1145/3579639}.
          Eine Sicherheitslücke kann \ac{zB} ein Passwort im Klartext in einer \textit{Manifest}-Datei sein~\cite{10.1145/3579639}.
    \item[Herausforderung 4]{- Abhängigkeiten zwischen Objekten\\}
          Die Abhängigkeit der Kubernetes Objekte untereinander stellt den Nutzer vor eine Herausforderung~\cite{dev-to-kubernetes-challenges,kubetools-io-kubernetes-manifest-management,spacelift-io-kubernetes-challenges,qovery-kubernetes-challenges}.
          Es kann \ac{zB} zu einer inkorrekten Konfiguration eines \textit{Labels} einer Ressource kommen~\cite{spacelift-io-kubernetes-challenges,qovery-kubernetes-challenges}.
          Dieses Problem kann auftreten, wenn der Nutzer nicht über alle verfügbaren \textit{Labels} informiert ist.
\end{description}

\section{Vorhandene Lösungen}

%Nachdem bestehende Herausforderungen beim Erstellen von Kubernetes \textit{Manifest} identifiziert wurden, konnte nach bestehenden Lösungen für diese gesucht werden.
%Die gefundenen Programme wurden anschließend auf fehlende Funktionalitäten zur Lösung der Herausforderungen untersucht.
Um bestehende Lösungen für die identifizierten Herausforderungen zu finden, wurde eine Websuche anhand folgender Suchbegriffe durchgeführt:
\begin{itemize}
    \setlength\itemsep{-0.5cm}
    \item ``kubernetes validation''
    \item ``kuberrnetes security validation''
    \item ``kubernetes autocompletion''
    \item ``kubernetes manifest tools''
    \item ``kubernetes manifest autocompletion''
    \item ``kubernetes IDE''
    \item ``kubernetes manifest editor''
\end{itemize}

Es wurden nur Programme berücksichtigt, für welche es in den letzten drei eine Aktualisierung gab.
Aus der Websuche ergaben sich neun Programme, die mindestens eine der Herausforderungen erfüllen und die
im definierten Zeitraum eine Aktualisierung erhalten haben.
Die Erfüllung einer Herausforderung wurde zuerst mit Hilfe der Quellen angenommen und anschließend
in einem Laborversuch verifiziert.
\\\\
Der Laborversuch wurde auf einem Rechner mit dem Betriebssystem Windows 11 durchgeführt.
Die Anwendung \textit{Docker for Desktop} wurde installiert~\cite{docker-for-desktop-overview}. Sie enthält ein Kubernetes Cluster enthält,
welches lokal auf einen einzigen \textit{Node} ausgeführt wird~\cite{docker-for-desktop-kubernetes}.
\\
Zuerst wurde ein \textit{Namespace} und ein \textit{Secret} auf dem Cluster mit folgenden Befehlen angelegt:
\begin{enumerate}
    \item
          \begin{minted}[linenos=false]{console}
        kubectl create namespace namespace-exp
    \end{minted}
    \item
          \begin{minted}[linenos=false]{console}
        kubectl create secret generic secret-exp --namespace=namespace-exp
    \end{minted}
\end{enumerate}
Es wurden zwei verschiedene \textit{Manifest}-Dateien erstellt,
die jeweils eine vollständige valide Definition eines \textit{Pods} und eines \textit{Deployments} enthalten.
Im Anhang befindet sich der Inhalt dieser Dateien.
\\
Um sicherzustellen, dass die Definition vollständig und valide ist, wurden die Objekte mit dem Kommandozeilentool \textit{kubectl} auf das Cluster ausgerollt.
Folgende Befehle wurden dazu genutzt:
\begin{enumerate}
    \item
          \begin{minted}[linenos=false]{console}
        kubectl create -f pod.yaml
    \end{minted}
    \item
          \begin{minted}[linenos=false]{console}
        kubectl create -f deployment.yaml
    \end{minted}
\end{enumerate}



Im nächsten Schritt wurden Testfälle definiert, die zur Beurteilung der Lösung einer Herausforderung herangezogen werden können.
\begin{description}
    \item[Testfall 1] falsche Einrückung(z.B Mapping-Schlüssel in Sequence definiert), fehlender Doppelpunkt
    \item[Testfall 2.1] fehlender Pflichtschlüssel, unbekannter Mapping-Schlüssel definiert, Scalar eines falschen Typs angegeben
    \item[Testfall 2.2] Autovervollständigung von Mapping-Schlüsseln
    \item[Testfall 3] Programm erkennt einen im Paper definierten Sicherheitsmangel
    \item[Testfall 4.1] nicht existierendes Kubernetes Objekt wurde bei der Definition angegeben. (z.B. secret, namespace)
    \item[Testfall 4.2] Autovervollständigung von Werten mit Hilfe von Cluster Ressourcen
\end{description}



%Lösung für 1 -> yaml syntax prüfung \\
%Lösung für 2 -> validation, autocompletion \\
%Lösung für 3 -> security validation \\
%Lösung für 4 -> übersicht der clusterressourcen und autocompletion \\

\begin{table}[h]
    \centering
    \begin{tabular}{lllll}
        \toprule
        \begin{tabular}{@{}l@{}} \emptycirc: nicht erfüllt \\ \halfcirc: teilweise erfüllt\\ \fullcirc: vollständig erfüllt \end{tabular} & \rotatebox{90}{Herausforderung 1} & \rotatebox{90}{Herausforderung 2} & \rotatebox{90}{Herausforderung 3} & \rotatebox{90}{Herausforderung 4} \\
        \midrule
        kubeconform~\cite{kubeconform-source}                                                                                             & \fullcirc                         & \halfcirc                         & \emptycirc                        & \emptycirc                        \\
        KubeLinter~\cite{kubelinter-source}                                                                                               & \fullcirc                         & \halfcirc                         & \emptycirc                        & \emptycirc                        \\
        kube-score~\cite{kubescore-source}                                                                                                & \fullcirc                         & \emptycirc                        & 3                                 & \emptycirc                        \\
        kubectl-validate~\cite{kubectl-validate-source}                                                                                   & \fullcirc                         & \halfcirc                         & \emptycirc                        & \emptycirc                        \\
        \midrule
        Lens~\cite{kubernetes-lens-source}                                                                                                & \fullcirc                         & \halfcirc                         & 3                                 & \halfcirc                         \\
        Monokle~\cite{monokle-source}                                                                                                     & \fullcirc                         & \fullcirc                         & \fullcirc                         & \halfcirc                         \\
        vscode-yaml~\cite{vscode-yaml-source}                                                                                             & \fullcirc                         & \fullcirc                         & \emptycirc                        & \emptycirc                        \\
        ValidCube~\cite{valid-cube-source}                                                                                                & \fullcirc                         & \halfcirc                         & \fullcirc                         & \emptycirc                        \\
        kpad~\cite{kpad-source}                                                                                                           & \fullcirc                         & \fullcirc                         & \emptycirc                        & \emptycirc                        \\
        \bottomrule
    \end{tabular}
    \caption{Zuordnungstabelle der Klassen der verschiedenen \acs{ast}-Implementierungen}
    \label{tbl:kubernetes-manifest-tools-capabilities}
\end{table}