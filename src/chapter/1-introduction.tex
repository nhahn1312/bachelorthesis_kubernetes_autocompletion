\chapter{Einleitung}\label{ch:introduction}

Die Skalierbarkeit und Ausfallsicherheit von IT-Systemen wird mit der zunehmenden Digitalisierung von Prozessen immer wichtiger.
Dies zeigt auch das Problem bei Erreichen des KfW-Portals am 26.09.23 für das Förderprogramm ``Solarstrom für Elektroautos''~\cite{kfw-not-reachable}.
Skalierbarkeit bezeichnet die Fähigkeit eines Systems bei erhöhter Nachfrage Ressourcen hinzuzufügen, um die gestiegene Nachfrage zu bewältigen und
diese Ressourcen bei sinkender Nachfrage wieder freizugeben. Das System kann seine Kapazität an wechselnde Anforderungen anpassen~\cite{it-system-scaling}.
Um die Skalierbarkeit von IT-Systemen zu realisieren, bietet sich die Software Kubernetes an.
Bei Kubernetes handelt es sich um eine Software zur Bereitstellung, Verwaltung und Skalierung von Anwendungen in Containern~\cite{kubernetes-overview}.
Ein Kubernetes Cluster besteht aus sogenannten Kubernetes Objekten, die den aktuellen Zustand des Clusters beschreiben. Dabei geben die Kubernetes Objekte an,
welche Anwendungen auf dem Cluster laufen, welche Ressourcen diesen zur Verfügung stehen und wie sich die Anwendungen, \acs{zB} im Fehlerfall, verhalten~\cite{kubernetes-objects}.

\section{Problemstellung}\label{sec:problem}

Die Definition von Kubernetes Objekten kann in deklarativen Konfigurationsdateien im \acs{yaml}-Format vorgenommen werden~\cite{kubernetes-config-declarative}.
Das Schreiben und Pflegen dieser Konfigurationsdateien ist schwierig und fehleranfällig, insbesondere wenn die Größe und Komplexität des Clusters zunimmt~\cite{kubernetes-config-problems} TODO: find more sources.
Im Rahmen dieser Arbeit werden die Möglichkeiten der Autovervollständigung und Validierung von Kubernetes Konfigurationsdateien untersucht.
Dabei werden folgende Fragen beantwortet:
\begin{enumerate}
    \item Wie kann die Autovervollständigung und Validierung von Kubernetes Konfigurationsdateien in eine \acs{ide} integriert werden?
    \item Welche Vor- und Nachteile hat die Implementierung im Vergleich zu bereits bestehenden Lösungen?
\end{enumerate}

\section{Methodik}
Um die in Kapitel~\ref{sec:problem} gestellten Fragen zu beantworten, wird eine \acs{ide}-Erweiterung entwickelt, welche die Validierung und Autovervollständigung von
Kubernetes Konfigurationsdateien ermöglicht. Anschließend wird die entwickelte Lösung mit bereits bestehenden Lösungen verglichen.
Hierbei werden ausgewählte Kriterien zur Beurteilung aus~\cite[A Large-Scale Study of Usability Criteria Addressed by Static Analysis Tools]{usability-criteria-static-analysis-tools} herangezogen.
Außerdem wird die Autovervollständigung ausgewählter Lösungen in einem Laborversuch untersucht.
Die Ergebnisse der Untersuchungen werden abschließend evaluiert.

\section{Aufbau der Arbeit}
Im folgenden Kapitel werden die technischen Grundlagen beschrieben. In Kapitel~\ref{ch:implementation} werden die Schritte zur Implementierung erläutert.
Der Vergleich der Implementierung mit ähnlichen Lösungen, sowie der Laborversuch erfolgen in Kapitel~\ref{ch:comparison}.
Die Vor- und Nachteile werden der verschiedenen Lösungen im anschließenden Kapitel evaluiert.
Abschließend folgt das Fazit der Arbeit und ein Ausblick auf weitere mögliche Schritte.