\chapter{Evaluation}\label{ch:evaluation}

Im vorliegenden Kapitel werden die Ergebnisse der Arbeit anhand der Zielsetzungen aus Kapitel~\ref{ch:introduction} zusammengefasst
und die daraus resultierenden Erkenntnisse beschrieben. Zuletzt wird mögliche weiterführende Forschung vorgestellt.

\section{Ergebnisse und Erkenntnisse}

Zunächst wird auf die in Kapitel~\ref{ch:introduction} definierten Zielsetzungen eingegangen und deren Ergebnisse zusammengefasst.
Die Erstellung von Kubernetes Konfigurationsdateien stellt ein Problem dar.
Die Arbeit sollte näher untersuchen, wie die Erstellung von Konfigurationsdateien vereinfacht werden kann.

\begin{enumerate}
    \item Welche Herausforderungen und Probleme treten beim Erstellen von Kubernetes Konfigurationsdateien auf?
\end{enumerate}
Die erste Frage zielte darauf ab die Probleme zu identifizieren. Bei einer Stichwortsuche nach wissenschaftlichen Arbeiten, konnten nur
fünf gefunden werden, die sich explizit mit Kubernetes Konfigurationsdateien
auseinandergesetzt haben.
Deshalb wurde eine Websuche mit zusätzlichen Suchbegriffen durchgeführt.
Diese ergab folgende Probleme und Herausforderungen:

\begin{itemize}
    \setlength\itemsep{-0.5cm}
    \item \textbf{Herausforderung 1}-\ac{yaml}-Syntax
    \item \textbf{Herausforderung 2}-Syntax der \textit{Manifest}-Dateien
    \item \textbf{Herausforderung 3}-Erkennen von Sicherheitslücken
    \item \textbf{Herausforderung 4}-Abhängigkeit von Cluster-Ressourcen
\end{itemize}
Nur Herausforderung 3 wird explizit in bereits bestehenden wissenschaftlichen Arbeiten untersucht~\cite{9476056,10.1145/3579639,10.1145/3468264.3473495}.
Alle anderen Herausforderungen konnten nicht anhand wissenschaftlicher Quellen verifiziert werden.
Es wurde gezeigt, dass die Zielsetzung nicht ausreichend in wissenschaftlichen Arbeiten untersucht wurde, die Herausforderungen jedoch vorhanden sind.
Dies zeigen zum einen die Ergebnisse der Websuche, aber auch die Existenz von bestehenden Lösungen für die Herausforderungen.

\begin{enumerate}
    \setcounter{enumi}{1}
    \item Welche Möglichkeiten zur Unterstützung bei der Erstellung von Konfigurationsdateien gibt es?
\end{enumerate}

Die der zweiten Zielsetzung wurde untersucht, welche Programme bereits existieren, um die Herausforderungen zu lösen.
Die Lösungen wurden mit Hilfe einer Websuche identifiziert.
Keine der Implementierungen konnte alle aus der ersten Zielsetzung resultierenden\ Herausforderungen lösen.
Dies wurde durch einen Laborversuch gezeigt.
Es stellte sich heraus, dass keine der vorhandenen Implementierungen eine Lösung für Herausforderung 4 aufweist.
\\\\
Die Suche nach bestehenden Lösungen bezog sich nur auf die Unterstützung beim Schreiben von Konfigurationsdateien.
Es wurden keine weiteren Möglichkeiten betrachtet, die den Nutzer, absehen vom Schreibvorgang selbst, unterstützen.
Beispiel hierfür wäre eine Übersicht des Clusters, die dem Nutzer keine aktive Unterstützung beim Schreiben bietet,
die Informationen jedoch den Schreibvorgang der Konfigurationsdateien vereinfachen können.
%- Alternative Lösung könnten Tools wie Helm genutzt werden (Kubernetes Manifest Templates)????
Die Ergebnisse des Laborversuchs sind nur unter den definierten Bedingungen gültig.
Sie waren dabei nicht für alle Programme optimal. Kommandozeilentools wurden durch die Aufnahme von Autovervollständigung als Testfall benachteiligt,
da sie diesen nicht erfüllen konnten.

\begin{enumerate}
    \setcounter{enumi}{2}
    \item Wie kann eine Lösung implementiert werden, die den Nutzer bei der Erstellung unterstützt?
\end{enumerate}

Die Implementierung einer eigenen Lösung konnte auf die Erkenntnisse aus der zweiten Zielsetzung aufbauen.
Herausforderung 4, die im Laborversuch keine der Anwendungen erfüllen konnte und Herausforderung 2, die zwei der Anwendungen teilweise erfüllt haben
wurden als Grundlage für die Anforderungen der Implementierung genutzt.
Die Anforderungen wurden Mit Hilfe der Testfälle aus dem Laborversuch für die beiden Herausforderungen definiert.
Es folgt eine Evaluation der einzelnen Anforderungen:
\begin{description}
    \item[Evaluation 1]{- Funktionalität für jedes Kubernetes Objekt\\}
          Die Autovervollständigung und Validierung wurde, wie in Anforderung 1 beschrieben umgesetzt.
          Die \ac{json}-Schema-Dateien bilden alle Objekte ab, da diese aus der \textit{OpenAPI}-Spezifikation generiert wurden.
    \item[Evaluation 2 bis 5]{- Autovervollständigung und Validierung \ac{json}-Schemas\\}
          Die Funktionalitäten konnten wie geplant umgesetzt werden. Das \ac{json}-Schema
          definiert Schlüsselwörter mit denen z.B. ein Pflichtschlüssel abgebildet werden kann.
    \item[Evaluation 6 und 7]{- Autovervollständigung und Validierung Cluster-Werte\\}
          Die Anforderungen werden von der implementierten Anwendung erfüllt.
          Eine nicht durch die Anforderungen vorgegebene Funktion ist die Autovervollständigung
          mittels Werten, die sich in den geöffneten Dokumenten befinden. Dies ermöglicht dem Nutzer eine Ressource, als Vorschlag zur Autovervollständigung
          zu erhalten, die sich noch nicht auf dem Cluster befindet.
          Durch die Klassenstruktur der Zuordnungsregeln, die in Kapitel~\ref{subsec:validation} eingeführt wurde, ist
          es möglich Autovervollständigung und Validierung für neue \textit{Mapping}-Schlüssel ohne hohen Aufwand zu implementieren.

\end{description}
Um die Erfüllung der Anforderungen zu verifizieren könnte der Laborversuch für die eigene Implementierung durchgeführt werden.
Der Einsatz des \ac{lsp} ermöglicht die Integration der Anwendung in andere \ac{ide}s wie z.B. IntelliJ.
Das Validieren und die Autovervollständigung mit Hilfe der \ac{json}-Schema-Dateien hat den Nachteil, dass diese für neue Kubernetes Versionen
erneut erstellt werden müssen.
Die Autovervollständigung und Validierung auf Basis der \textit{KubernetesAPI} funktioniert außerdem nur für ausgewählte Ressourcen.

\begin{enumerate}
    \setcounter{enumi}{3}
    \item Welche Vor- und Nachteile hat die Implementierung im Vergleich zu bereits bestehenden Lösungen?
\end{enumerate}

Die Vor- und Nachteile ergeben sich aus den Ergebnissen von Kapitel~\ref{ch:preparation-analysis} und Kapitel~\ref{ch:comparison}.
Kapitel~\ref{ch:preparation-analysis} hat die fehlenden Funktionen von vorhandenen Anwendungen aufgezeigt. Ein Vorteil der eigenen Implementierung ist
somit der größere Funktionsumfang bei der Unterstützung vom Schreiben der Konfigurationsdateien.
Die Anwendung bietet Autovervollständigung und Validierung für ausgewählte Cluster-Ressourcen in den \textit{Manifest}-Dateien an, was im Laborversuch
mit keiner anderen Lösung möglich war. Allerdings unterstützt die Anwendung den Nutzer nicht beim Erkennen von Sicherheitsmängeln in
den Konfigurationsdateien. Ein Vorteil gegenüber den bestehenden
Kommandozeilentools, ist die \ac{ide}-Integration der eigenen Implementierung, die dem Nutzer eine Integration in den Arbeitsablauf ermöglicht~\cite{usability-criteria-static-analysis-tools}.
Der Vergleich in Kapitel~\ref{ch:comparison} zeigt Nachteile der eigenen Implementierung. Das Programm kann nur $19\%$ der Kriterien, die die
Benutzerfreundlichkeit von statischen Code-Analysetools beurteilen, erfüllen. Im Vergleich erfüllen nur drei der neun Programme weniger Kriterien.
Keines der Programme konnte beim Vergleich mehr als $50\%$ der Kriterien erfüllen.
Monokle konnte mit $42\%$ die meisten Kriterien von allen untersuchten Anwendungen erfüllen.
\\\\
Daraus lässt sich schlussfolgern, dass die Kriterien entweder sehr umfangreich sind oder dass die vorhanden Anwendungen
die Benutzerfreundlichkeit nicht als wichtiges Kriterium ansehen.
Weiterhin könnte argumentiert werden, dass die für den Vergleich herangezogene Arbeit ausschließlich Programme zur statischen Analyse
von Quellcode untersucht hat. Quellcode von Programmiersprachen hat eine höhere Komplexität, als die \ac{yaml}-Syntax von Kubernetes.
Um benutzerfreundlich zu sein, muss ein statisches Analysetool für Quellcode daher mehr Anforderungen erfüllen.

\section{Weiterführende Forschung}

Mit der vorliegenden Arbeit konnten erste Erkenntnisse zu Problemen und Herausforderungen beim Schreiben Kubernetes
Konfigurationsdateien gewonnen werden. Zukünftige Forschungen könnten die Probleme und Herausforderungen
anhand einer empirischen Studie untersuchen. Dabei könnten zusätzlich Kriterien zur Verifikation von Lösungen zu
den Herausforderungen entworfen werden, die über einen Laborversuch hinaus gehen.
\\
Ein wichtiger Aspekt könnte auch die Implementierung von statischen Analysetools für die Konfiguration von
verteilten Systemen sein. Die Implementierung aus Kapitel~\ref{ch:implementation} bietet Autovervollständigung und Validierung
der Cluster-Ressourcen nur für eine begrenzte Auswahl von Ressourcen an. Es könnte demnach untersucht werden,
wie eine Anwendung entworfen werden kann, die flexibel für die Konfiguration von verteilten Systemen eingesetzt werden kann.
Dies könnte z.B. mit Einführung eines Standards möglich sein.