\chapter{Fazit}\label{ch:conclusion}
Im folgenden Kapitel werden die Ergebnisse der Arbeit zusammengefasst. Im letzten Abschnitt
der Arbeit folgt ein Ausblick auf weiterführende Forschung.

\section{Zusammenfassung}

In der vorliegenden Arbeit wurden Probleme und Herausforderungen beim Schreiben
von Kubernetes Konfigurationsdateien herausgearbeitet.
Die vier Herausforderungen sind: Die \ac{yaml}-Syntax, die Syntax von Kubernetes \textit{Manifest}-Dateien,
das Erkennen von Sicherheitslücken beim Schreiben der Dateien und die Abhängigkeiten zwischen Objekten im Cluster.
Dabei wurden nur die Sicherheitslücken in \textit{Manifest}-Dateien bereits in wissenschaftlichen Arbeiten behandelt.
Alle übrigen Herausforderungen wurden mit einer Websuche identifiziert.
\\
Weiterhin wurden vorhandene Lösungen für die Herausforderungen gefunden.
Dazu wurde ein Laborversuch durchgeführt, der erfassen sollte, welche Anwendung welche Herausforderung erfüllt.
Dabei stellte sich heraus, dass keine der Lösungen die Abhängigkeiten zwischen Objekten im Cluster
für den Nutzer validieren oder autovervollständigen kann.
\\
Im nächsten Schritt wurde eine Lösung zur Validierung und Autovervollständigung von ausgewählten
Ressourcen aus dem Cluster implementiert. Dazu wurde das \ac{lsp} genutzt, welches die
\ac{ide}-Integration der Lösung ermöglicht.
Die Anforderungen an die Anwendung ergaben sich aus den Testfällen des Laborversuchs.
\\
Zuletzt wurde die eigene Implementierung mit den vorhandenen Lösungen auf Basis
von Kriterien zur Benutzerfreundlichkeit von statischen Code-Analysetools verglichen, die aus einer wissenschaftlichen Arbeit herausgearbeitet
wurden. Der Vergleich hat gezeigt, dass die eigene Lösung nur $19\%$ der Kriterien erfüllt.
Nur drei von neun anderen Lösungen konnten weniger Kriterien erfüllen.
Die beste Lösung erfüllt $42\%$ der Kriterien.
Das Ergebnis zeigt, dass die Kriterien entweder sehr umfangreich sind oder die Benutzerfreundlichkeit
keine wichtige Rolle spielt.
\\
Die Bewertung der Anforderungen der Anwendung hat gezeigt, dass diese alle Anforderungen erfüllen konnte.
Dies wurde für alle Anforderungen, außer Anforderung für 1, mit Hilfe von Modultests sichergestellt.
Das Erfüllen von Anforderung 1, welche die Funktionalität für alle Kubernetes Objekte fordert, wurde angenommen.
Die \ac{json}-Schema-Dateien die Anwendung nutzt, wurden aus der \textit{OpenAPI}-Spezifikation der \textit{KubernetesAPI} generiert
und diese sollte Schnittstellenbeschreibungen für alle Objekte enthalten.
Der Praxistest der Anwendung hat gezeigt, dass die Validierung der Anwendung für Cluster-Ressourcen
funktioniert, aber noch Verbesserungen benötigt.

\section{Ausblick}

In dieser Arbeit wurde gezeigt, dass es wenige wissenschaftliche Arbeiten zur Erstellung von
Kubernetes Konfigurationsdateien und den damit verbunden Problemen und Herausforderungen gibt.
Die Probleme und Herausforderungen jedoch existieren, was zum einen die Websuche und zum anderen die
Existenz von Anwendungen zur Lösung der Probleme zeigt. Zukünftige Forschungen könnten die Probleme und Herausforderungen
anhand einer empirischen Studie untersuchen.
\\
Es hat sich außerdem herausgestellt, dass bestehende Lösungen einige Funktionslücken aufweisen.
Dies konnte in einem Laborversuch gezeigt werden, welcher aber nicht zur Verifikation herangezogen werden kann.
Die Testfälle des Laborversuchs wurden, bis auf die Testfälle für die Sicherheitsmängel,
mit persönlichem Wissen über die \textit{Manifest}-Dateien und Herausforderungen erstellt.
Das heißt, es müssten belegbare Kriterien für das Erfüllen einer Herausforderung entwickelt werden.
\\
Ein wichtiger Aspekt könnte auch die Implementierung von statischen Analysetools für die Konfiguration von
verteilten Systemen sein. Die Implementierung aus Kapitel~\ref{ch:implementation} bietet Autovervollständigung und Validierung
der Cluster-Ressourcen nur für eine begrenzte Auswahl von Ressourcen an. Es könnte demnach untersucht werden,
wie eine Anwendung entworfen werden kann, die flexibel für die Konfiguration von verteilten Systemen eingesetzt werden kann.
Dies könnte z.B. mit Einführung eines Standards möglich sein.
\\
Der Vergleich auf Basis der Kriterien zur Benutzerfreundlichkeit von statischen Code-Analysetools hat gezeigt,
dass die implementierte Anwendung nicht benutzerfreundlich ist.
In Zukunft könnte dies geändert werden, indem Lösungen für die nicht erfüllten Kriterien implementiert werden.
\\
Um die Vor- und Nachteile der Implementierung beim Praxiseinsatz eingehender zu untersuchen, könnte eine Studie mit
zwei Gruppen durchgeführt werden. Dabei erhält die eine Gruppe die implementierte Anwendung und die andere Gruppe
eine bereits bestehende Lösung. Um die Erfahrungen der Personen zu erfassen, könnte ein Fragenkatalog erarbeitet werden.
Eine Bewertung könnte dann anhand der Antworten durchgeführt werden.