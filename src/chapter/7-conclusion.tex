\chapter{Fazit}\label{ch:conclusion}
In der vorliegenden Arbeit wurden Probleme und Herausforderungen beim Schreiben
von Kubernetes Konfigurationsdateien herausgearbeitet.
Die vier Herausforderungen sind: Die \ac{yaml}-Syntax, die Syntax von Kubernetes \textit{Manifest}-Dateien,
das Erkennen von Sicherheitslücken beim Schreiben der Dateien und die Abhängigkeiten zwischen Objekten im Cluster.
Dabei wurden nur die Sicherheitslücken in \textit{Manifest}-Dateien bereits in wissenschaftlichen Arbeiten behandelt.
Alle übrigen Herausforderungen wurden mit einer Websuche identifiziert.
\\
Weiterhin wurden vorhandene Lösungen für die Herausforderungen gefunden.
Dazu wurde ein Laborversuch durchgeführt, der erfassen sollte, welche Anwendung welche Herausforderung erfüllt.
Dabei stellte sich heraus, dass keine der Lösungen die Abhängigkeiten zwischen Objekten im Cluster
für den Nutzer validieren oder autovervollständigen kann.
\\
Im nächsten Schritt wurde eine Lösung zur Validierung und Autovervollständigung von ausgewählten
Ressourcen aus dem Cluster implementiert. Dazu wurde das \ac{lsp} genutzt, welches die
\ac{ide}-Integration der Lösung ermöglicht.
Die Anforderungen an die Anwendung ergaben sich aus den Testfällen des Laborversuchs.
Die Anwendung konnte alle gestellten Anforderungen erfüllen.
\\
Zuletzt wurde die eigene Implementierung mit den vorhandenen Lösungen auf Basis
von Kriterien zur Benutzerfreundlichkeit von statischen Code-Analysetools verglichen, die aus einer wissenschaftlichen Arbeit herausgearbeitet
wurden. Der Vergleich hat gezeigt, dass die eigene Lösung nur $19\%$ der Kriterien erfüllt.
Nur drei von neun anderen Lösungen konnten weniger Kriterien erfüllen.
Die beste Lösung erfüllt $42\%$ der Kriterien.
Das Ergebnis zeigt, dass die Kriterien entweder sehr umfangreich sind oder die Benutzerfreundlichkeit
keine wichtige Rolle spielt.
\\
In dieser Arbeit wurde gezeigt, dass es wenige wissenschaftliche Arbeiten zur Erstellung von
Kubernetes Konfigurationsdateien und den damit verbunden Problemen und Herausforderungen gibt.
Die Probleme und Herausforderung jedoch existieren, was zum einen die Websuche und zum anderen die
Existenz von Anwendungen zur Lösung der Probleme zeigt. Es könnte hier in Zukunft eine
empirische Studie zu den Problemen und Herausforderungen durchgeführt werden.
Es hat sich außerdem herausgestellt, dass bestehende Lösungen einige Funktionslücken aufweisen.
Dies konnte in einem Laborversuch gezeigt werden, welcher aber nicht zur Verifikation herangezogen werden kann.
Die Testfälle des Laborversuchs wurden, bis auf die Testfälle für die Sicherheitsmängel,
mit persönlichem Wissen über die \textit{Manifest}-Dateien und Herausforderungen erstellt.
Das heißt, es müssten belegbare Kriterien für das Erfüllen einer Herausforderung entwickelt werden.
Die Anwendung, die im Rahmen dieser Arbeit entwickelt wurde, hat gezeigt, dass es möglich ist
Validierung und Autovervollständigung für Kubernetes Konfigurationsdateien, auf Basis von Cluster-Ressourcen, zu ermöglichen.
Der aktuelle Stand der Implementierung ist allerdings nicht benutzerfreundlich.
In Zukunft könnte dies geändert werden, indem man Lösungen für die nicht erfüllten Kriterien implementiert.