\chapter{Vergleich}\label{ch:comparison}

In dem vorliegenden Kapitel werden die vorhandenen Lösungen aus Kapitel~\ref{ch:preparation-analysis} mit
der Implementierung aus Kapitel~\ref{ch:implementation} unter Berücksichtigung der Kriterien aus der wissenschaftlichen Arbeit
~\cite[A Large-Scale Study of Usability Criteria Addressed by Static Analysis Tools]{usability-criteria-static-analysis-tools} verglichen.
Die Arbeit beschäftigt sich mit Kriterien zur Benutzerfreundlichkeit von statischen Code-Analysetools und führt anhand dieser einen
Vergleich durch. Die Kriterien werden in diesem Kapitel herausgearbeitet. Anschließend wird entschieden, welche der Kriterien
im Kontext dieser Arbeit für den Vergleich genutzt werden. Zuletzt folgen die Ergebnisse des Vergleichs mit den ausgewählten Kriterien.

\section{Auswahl der Kriterien}

Die Kriterien wurden als Erstes in sechs Kategorien eingeteilt, die den Unterkapiteln aus dem dritten Kapitel der wissenschaftlichen Arbeit
entsprechen. Es folgt eine Beschreibung der einzelnen Kategorien~\cite{usability-criteria-static-analysis-tools}:

\begin{description}
    \setlength\itemsep{-0.5cm}
    \item[Kategorie 1]{- Warnmeldungen\\}
          Die Kriterien dieser Kategorie prüfen die Warnmeldungen der Lösungen und wie diese dem Nutzer beschrieben werden.
    \item[Kategorie 2]{- Unterstützung bei der Fehlerbehebung\\}
          In dieser Kategorie befinden sich Kriterien, die beurteilen, ob und wie der Nutzer bei der Fehlerbehebung unterstützt wird.
    \item[Kategorie 3]{- falsch-positive Fehler\\}
          Diese Kategorie bewertet, welche Mechanismen statische Code-Analysetools bieten, um falsch-positive Fehler zu unterdrücken
          und den Nutzer dabei zu unterstützen falsch-positive Fehler zu erkennen.
    \item[Kategorie 4]{- Rückmeldungen vom Nutzer\\}
          Die Kriterien der Kategorie beurteilen, ob und wie sich die Lösung an die Bedürfnisse des Nutzers anpassen lässt.
          Darunter fällt z.B. die Möglichkeit, spezifische Regeln der Analyse zu deaktivieren.
    \item[Kategorie 5]{- Integration in den Arbeitsablauf\\}
          Die Kriterien dieser Kategorie prüfen, wie sich die Lösung in den Arbeitsablauf des Nutzers Integrieren lässt.
    \item[Kategorie 6]{- Benutzeroberfläche\\}
          Die Kriterien zur Benutzeroberfläche untersuchen, wie Fehlermeldungen dem Nutzer auf der Oberfläche präsentiert werden.
\end{description}

Die Tabellen \ref{tbl:criteria-category-1}, \ref{tbl:criteria-category-2}, \ref{tbl:criteria-category-3}, \ref{tbl:criteria-category-4}, \ref{tbl:criteria-category-5}  und \ref{tbl:criteria-category-6}
zeigen eine Übersicht aller Kriterien mit deren einer Kurzbeschreibung und ob diese für den Vergleich im Rahmen der Arbeit ausgewählt wurden.
Sollte ein Kriterium nicht ausgewählt sein, wird eine Begründung angegeben. Für jede Kategorie wurde eine eigene Tabelle erstellt.

\begin{table}[htp]
    \centering
    \begin{tabularx}{\columnwidth}{XXl}
        \toprule
        \textbf{Name}              & \textbf{Beschreibung}                                                                                                                & \textbf{ja/nein} \\
        \midrule
        Interne Gründe             & Der Nutzer erhält eine Beschreibung der Gründe der Warnmeldung.                                                                      & ja               \\
        \midrule
        Auslöser des Fehlers       & Der Nutzer kann verfolgen, was den beschriebenen Fehler ausgelöst hat.                                                               & nein$^{(1)}$     \\
        \midrule
        Folgen des Fehlers         & Die Folgen eines Fehlers werden erläutert.                                                                                           & ja               \\
        \midrule
        Beispiele zum Fehler       & Die Warnmeldung wird anhand eines Beispiels näher erläutert.                                                                         & ja               \\
        \midrule
        Weitere Informationen      & Dem Nutzer werden weitere Informationen zu einem Fehler angeboten, z.B. der Link zu einer Website.                                   & ja               \\
        \midrule
        Kontext zum Fehler         & Beurteilt, ob die Lösung kontextbezogene Warnmeldungen anbietet oder nur generische Warnmeldungen mit Austausch von Variablen nutzt. & ja               \\
        \midrule
        Fehlercode                 & Untersucht, ob eine Anwendung einen einzigartigen Fehlercode für einen Fehler bereitstellt.                                          & ja               \\
        \midrule
        Grundlegende Informationen & Prüft, ob die Lösung eine Kategorisierung und Lokalisierung des Fehlers ermöglicht und ob es eine Einteilung nach Schweregrad gibt.  & ja               \\
        \bottomrule
    \end{tabularx}
    \caption{Kriterien der Kategorie 1}
    \label{tbl:criteria-category-1}
\end{table}

\begin{enumerate}[label= (\arabic*)]
    \item Dieses Kriterium ist nur für statische Analysetools für Programmiersprachen relevant. Dabei geht es darum den Auslöser eines Fehlers zu finden.
          Bei dem Quelltext einer Programmiersprache wäre dies z.B. die Methode, die den Fehler verursacht hat.
          Die Fehler in \textit{Manifest}-Dateien sind direkt lokalisierbar.
\end{enumerate}

\FloatBarrier

\begin{table}[htp]
    \centering
    \begin{tabularx}{\columnwidth}{lXl}
        \toprule
        \textbf{Name}                & \textbf{Beschreibung}                                                                       & \textbf{ja/nein} \\
        \midrule
        Schnelle Fehlerbehebung      & Die Anwendung gibt einen Vorschlag zum Beheben des Fehlers an.                              & ja               \\
        \midrule
        Alternative Lösungen         & Die Anwendung bietet alternative Lösungen zur Fehlerbehebung an.                            & ja               \\
        \midrule
        Vorschau zur Fehlerbehebung  & Dem Nutzer wird in einer Vorschau gezeigt, wie die Datei nach Fehlerbehebung aufgebaut ist. & ja               \\
        \midrule
        Beispiele zur Fehlerbehebung & Der Nutzer erhält Beispiele zur Lösung des Fehlers.                                         & ja               \\
        \midrule
        Anleitung zur Fehlerbehebung & Die Anwendung zeigt dem Nutzer alle nötigen Schritte zur Fehlerbehebung.                    & ja               \\
        \bottomrule
    \end{tabularx}
    \caption{Kriterien der Kategorie 2}
    \label{tbl:criteria-category-2}
\end{table}

\begin{table}[htp]
    \centering
    \begin{tabularx}{\columnwidth}{lXl}
        \toprule
        \textbf{Name}               & \textbf{Beschreibung}                                                                                                                                         & \textbf{ja/nein} \\
        \midrule
        Ausblenden von Fehlern      & Die Anwendung gibt dem Nutzer Möglichkeit eine Warnmeldung als falsch-positiv zu markieren und auszublenden.                                                  & ja               \\
        \midrule
        Punktesystem                & Die Anwendung fügt Warnmeldungen ein Punktesystem hinzu, mit dessen Hilfe der Nutzer abschätzen kann, ob es sich um einen falsch-positiven Fehler handelt.    & ja               \\
        \midrule
        Nutzerwissen zur Einordnung & Bei fehlendem Kontext fragt die Anwendung den Nutzer, um den Fehler besser einzuordnen und die Nutzereingabe zur Verbesserung der Warnmeldungen zu verwenden. & ja               \\
        \bottomrule
    \end{tabularx}
    \caption{Kriterien der Kategorie 3}
    \label{tbl:criteria-category-3}
\end{table}


\begin{table}[htp]
    \centering
    \begin{tabularx}{\columnwidth}{lXl}
        \toprule
        \textbf{Name}            & \textbf{Beschreibung}                                                                                                                 & \textbf{ja/nein} \\
        \midrule
        Anpassung der Regeln     & Der Nutzer kann die Regeln zur Validierung anpassen und bei Bedarf bestimmte Regeln deaktivieren.                                     & ja               \\
        \midrule
        Verarbeitung von Fehlern & Die Anwendung nutzt Fehler, die der Nutzer bereits behoben hat, um diese besser von falsch-positiven Fehlern unterscheiden zu können. & ja               \\
        \midrule
        temporäres Ausblenden    & Der Nutzer hat die Möglichkeit einen Fehler temporär auszublenden, z.B. durch eine Annotation.                                        & ja               \\
        \midrule
        Filter zum Ausblenden    & Die Anwendung ermöglicht dem Nutzer das Ausblenden von Warnmeldungen mit Hilfe von Filtern                                            & ja               \\
        \bottomrule
    \end{tabularx}
    \caption{Kriterien der Kategorie 4}
    \label{tbl:criteria-category-4}
\end{table}

\begin{table}[t]
    \centering
    \begin{tabularx}{\columnwidth}{XXl}
        \toprule
        \textbf{Name}                    & \textbf{Beschreibung}                                                              & \textbf{ja/nein} \\
        \midrule
        Priorisierung von Fehlern        & Die Anwendung priorisiert die Fehler automatisch für den Nutzer                    & ja               \\
        \midrule
        \ac{ide}-Integration             & Die Anwendung kann mit einer Erweiterung in eine \ac{ide} integriert werden        & ja$^{(1)}$       \\
        \midrule
        Alleinstehendes Programm         & Bei der Anwendung handelt es sich um ein alleinstehendes installierbares Programm. & nein$^{(2)}$     \\
        \midrule
        Browser-Integration              & Die Anwendung ist im Browser ausführbar                                            & nein$^{(3)}$     \\
        \midrule
        Integration in den Arbeitsablauf & Prüft, ob das Programm in den Arbeitsablauf integriert werden kann.                & nein$^{(4)}$     \\
        \midrule
        Laufzeit                         & Messung der Laufzeit der Lösungen                                                  & nein$^{(5)}$     \\
        \midrule
        Rückmeldung bei Fehlerbehebung   & Der Nutzer erhält von der Anwendung Rückmeldung, dass ein Fehler behoben wurde     & ja               \\
        \bottomrule
    \end{tabularx}
    \caption{Kriterien der Kategorie 5}
    \label{tbl:criteria-category-5}
\end{table}

\FloatBarrier

\begin{enumerate}[label= (\arabic*)]
    \item Die wissenschaftliche Arbeit gibt an, dass statische Analysetools zur besseren Integration in den Arbeitsablauf in eine \ac{ide} integriert sein sollten.
          Damit handelt es sich um ein qualitatives Kriterium.
    \item Die Kriterien sollen zur qualitativen Beurteilung herangezogen werden. Dieses Kriterium gibt an, um welche Art von Programm es sich handelt.
    \item Siehe (2)
    \item Die wissenschaftliche Arbeit gibt hier an, dass sich dieses Kriterium vollständig mit ``IDE-Integration'' überdeckt. Es wurde letztlich nur überprüft, ob die Anwendung in eine \ac{ide} integriert werden kann.
    \item Es gibt keine zuverlässige Methode, um einen Unterschied in der Laufzeit bei den zu vergleichenden Lösungen festzustellen.
\end{enumerate}

\begin{table}[t]
    \centering
    \begin{tabularx}{\columnwidth}{XXl}
        \toprule
        \textbf{Name}                         & \textbf{Beschreibung}                                                                                                                                                       & \textbf{ja/nein} \\
        \midrule
        Markieren von Fehlern und Warnsymbole & Die Anwendung markiert den Fehler in der Datei und zeigt Warnsymbole an.                                                                                                    & ja               \\
        \midrule
        Erfüllen von Erwartungen              & Die Anwendung verhält sich den Erwartungen des Nutzers entsprechend. Es wird z.B. überprüft, ob sich ein Kommandozeilentool ähnlich wie andere Kommandozeilentools verhält. & nein$^{(1)}$     \\
        \midrule
        Übersicht von Fehlermeldungen         & Die Anwendung stellt eine Übersicht aller Fehler bereit.                                                                                                                    & ja               \\
        \midrule
        Historie der bereits behobenen Fehler & Der Nutzer kann in der Anwendung eine Historie der bereits behobenen Fehler sehen und so seinen Fortschritt überwachen.                                                     & ja               \\
        \midrule
        Suche von Fehlern                     & Die Anwendung ermöglicht die Suche in allen vorhandenen Warnmeldungen                                                                                                       & ja               \\
        \bottomrule
    \end{tabularx}
    \caption{Kriterien der Kategorie 6}
    \label{tbl:criteria-category-6}
\end{table}

\FloatBarrier

\begin{enumerate}[label= (\arabic*)]
    \item Es fehlen Kriterien zur Beurteilung, wann sich ein Programm wie ein ähnliches verhält.
\end{enumerate}

\vspace{-1cm}

\section{Ergebnisse}

\vspace{-0.5cm}

Die Beurteilung der ausgewählten Kriterien wurde mit Hilfe der Dateien aus dem Laborversuch aus Kapitel~\ref{ch:preparation-analysis} und der offiziellen Dokumentation der Programme vorgenommen.
Die folgende Tabelle zeigt die Ergebnisse des Vergleichs. Dabei ist zu sehen, wie viele Kriterien einer Kategorie eine Anwendung prozentual erfüllt hat. Außerdem ist der Gesamtanteil für jede Anwendung dargestellt.
Eine detaillierte Auskunft über die erfüllten Kriterien befindet sich im Anhang.

\begin{table}[htp]
    \centering
    \begin{tabular}{llllllll}
        \toprule
        \textbf{Anwendung}      & \rotatebox{90}{Kategorie 1} & \rotatebox{90}{Kategorie 2} & \rotatebox{90}{Kategorie 3} & \rotatebox{90}{Kategorie 4} & \rotatebox{90}{Kategorie 5} & \rotatebox{90}{Kategorie 6} & \rotatebox{90}{Gesamt} \\
        \midrule
        kubeconform             & $29\%$                      & $0\%$                       & $0\%$                       & $50\%$                      & $0\%$                       & $0\%$                       & $15\%$                 \\
        KubeLinter              & $43\%$                      & $20\%$                      & $0\%$                       & $50\%$                      & $0\%$                       & $0\%$                       & $23\%$                 \\
        kube-score              & $57\%$                      & $0\%$                       & $0\%$                       & $50\%$                      & $0\%$                       & $0\%$                       & $23\%$                 \\
        kubectl-validate        & $29\%$                      & $0\%$                       & $0\%$                       & $0\%$                       & $0\%$                       & $0\%$                       & $8\%$                  \\
        \midrule
        Lens                    & $0\%$                       & $0\%$                       & $0\%$                       & $25\%$                      & $0\%$                       & $0\%$                       & $4\%$                  \\
        Monokle                 & $71\%$                      & $0\%$                       & $0\%$                       & $25\%$                      & $66\%$                      & $75\%$                      & $42\%$                 \\
        vscode-kubernetes-tools & $43\%$                      & $0\%$                       & $0\%$                       & $25\%$                      & $33\%$                      & $75\%$                      & $26\%$                 \\
        ValidCube               & $57\%$                      & $20\%$                      & $0\%$                       & $0\%$                       & $0\%$                       & $0\%$                       & $19\%$                 \\
        Eigene Implementierung  & $14\%$                      & $0\%$                       & $0\%$                       & $0\%$                       & $33\%$                      & $75\%$                      & $19\%$                 \\
        \bottomrule
    \end{tabular}
    \caption{Ergebnisse des Vergleichs}
    \label{tbl:kubernetes-manifest-tools-comparison}
\end{table}
