\chapter{Technische Grundlagen}\label{ch:technical-basics}

\section{Virtualisierung}

Mit Hilfe von Virtualisierung lassen sich mehrere simulierte Umgebungen aus einem einzigen Hardwaresystem, im folgenden \textit{Host} genannt, erstellen.
Dazu wird Software genutzt, die die physische Hardware emuliert~\cite{virtualization-techtarget}. Diese Software wird auch \textit{Hypervisor} genannt.
Der \textit{Hypervisor} abstrahiert die Hardware Ressourcen des \textit{Hosts} und stellt diese als einen gemeinsamen Pool zur Verfügung.
Die physische Hardware des \textit{Hosts} kann somit in mehrere isolierte Umgebungen, auch virtuelle Maschinen genannt, aufgeteilt werden.
Mithilfe des \textit{Hypervisors} können mehrere unterschiedliche Betriebssysteme parallel ausgeführt werden~\cite{virtualization-red-hat, hypervisor-red-hat}.
Das heißt, dass jede virtuelle Maschine ein eigenes Betriebssystem benötigt, um eine Anwendung auszuführen.

Sollen zwei verschiedene Anwendungen isoliert voneinander ausgeführt werden, werden dafür zwei virtuelle Maschinen mit jeweils einem eigenen Betriebssystem benötigt.
Jede virtuelle Maschine führt dabei ihren eigenen Satz von Systemprozessen aus, was neben der Ausführung der Anwendung zusätzliche Ressourcen erfordert~\cite{kubernetes-in-action-introduction}.
Um dieses Problem zu lösen wurde die Containervirtualisierung eingeführt.

\begin{figure}
      \centering
      \begin{subfigure}{.5\textwidth}
            \centering
            \includesvg{images/full-virtualization.svg}
            \caption{Vollvirtualisierung}
            \label{fig:full-virtualization}
      \end{subfigure}%
      \begin{subfigure}{.5\textwidth}
            \centering
            \includesvg{images/container-virtualization.svg}
            \caption{Containervirtualisierung}
            \label{fig:container-virtualization}
      \end{subfigure}
      \caption{Vergleich Voll- und Containervirtualisierung}
      \label{fig:virtualization-comparison}
\end{figure}

Abbildung~\ref{fig:virtualization-comparison} zeigt einen Vergleich zwischen der soeben eingeführten Virtualisierung mittels \textit{Hypervisor} und
der Containervirtualisierung.
Abbildung~\ref{fig:full-virtualization} zeigt die Komponenten einer Vollvirtualisierung, bei der der \textit{Hypervisor}
eine auf dem \textit{Host}-Betriebssystem installierte Software ist.
\\
Die Komponenten einer Containervirtualisierung sind auf Abbildung~\ref{fig:container-virtualization} dargestellt.
Im Gegensatz zur Vollvirtualisierung können Anwendungen über sogenannte \textit{Container} voneinander getrennt werden.
Ein \textit{Container} ist ein einziger, isolierter Prozess, der im Betriebssystem des Hosts läuft und nur die Ressourcen nutzt, die von der Anwendung benötigt werden.
Es entsteht so kein Mehraufwand für Systemprozesse, wie bei der Vollvirtualisierung~\cite{kubernetes-in-action-introduction}.
Um die Bereitstellung von \textit{Containern} zu ermöglichen wird eine \textit{Container Runtime} benötigt~\cite{kubernetes-overview}.
\\
Eine \textit{Container Runtime} ist eine Softwarekomponente, die die Ausführung von \textit{Containern} auf einem Host-Betriebssystem ermöglicht.
Die Aufgaben einer \textit{Container Runtime} sind dabei das Erstellen, Starten, Stoppen und Löschen von \textit{Containern},
das Überwachen der Systemressourcen des Hosts, das Isolieren von Systemressourcen für einen \textit{Container}
und das Laden von \textit{Container Images} aus einem \textit{Respository}~\cite{container-runtime-aquasec}.
\\
Ein \textit{Container Image} enthält Software und alle Ressourcen, die die Software zur Ausführung benötigt.
Neben dem Programmcode und den Bibliotheken, die die Anwendung zur Laufzeit nutzt, können dies auch weitere Programme sein, auf die zugegriffen werden muss.
\textit{Container Images} werden über ein zentrales \textit{Respository} verwaltet und können darüber bei Bedarf heruntergeladen werden~\cite{container-image-dev-insider}.

\section{Kubernetes}

Kubernetes (K8s) wurde ursprünglich von Google entwickelt und 2014 als Open-Source-Software veröffentlicht.
Kubernetes ist eine Software zur Bereitstellung, Verwaltung und Skalierung von Anwendungen in Containern. Der Name kommt aus dem Griechischen und bedeutet Steuermann.
Die Abkürzung K8s setzt sich aus dem ersten und letzten Buchstaben von Kubernetes, sowie der Anzahl der Buchstaben die diese voneinander trennen, zusammen~\cite{kubernetes-google-cloud}.

\subsection{Architektur}

Kubernetes abstrahiert die zur Verfügung stehende Hardware und fasst mehrere physische oder virtuelle Maschinen,
die als \textit{Nodes} bezeichnet werden, zu einem Cluster zusammen.
Die zugrundeliegende Infrastruktur wird dabei ausgeblendet und Kubernetes betrachtet die \textit{Nodes} als eine einzige, große Ressource.
\\
Jedes Cluster besteht aus verschiedenen Typen von \textit{Nodes}.
Für ein funktionierendes Kubernetes Cluster wird mindestens ein \textit{Master Node (Control Plane)} und ein \textit{Worker Node} benötigt~\cite{kubernetes-in-action-introduction}.
Auf Abbildung~\ref{fig:kubernetes-architecture} ist der minimale Aufbau eines Kubernetes-Clusters schematisch dargestellt. Es folgt eine
Beschreibung der Aufgaben des \textit{Master} und \textit{Worker Nodes}, sowie eine Beschreibung der einzelnen Komponenten.

\begin{figure}[htp] % htp = hier (h), top (t), oder auf einer eigenen Seite (p).
      \centering
      \includesvg[width=0.5\textwidth]{images/kubernetes-architecture.svg} % width immer angeben!
      \caption{Komponenten einer Kubernetes-Clusters}
      \label{fig:kubernetes-architecture}
\end{figure}

Das \textit{Master Node} ist für die Administration des Clusters zuständig. Es koordiniert und überwacht die \textit{Worker Nodes} des Clusters.
Seine Aufgaben sind unter anderem die Planung und Skalierung von Anwendungen und die Aufrechterhaltung des Zustands des Clusters~\cite{kubernetes-cluster-vmware}.
Das \textit{Master Node} besteht aus den vier folgenden Komponenten:
\begin{description}
      \item[API-Server]
            Der \textit{API-Server} koordiniert alle Prozesse zwischen dem \textit{Master Node} und den Komponenten der \textit{Worker Nodes}.
            Er stellt außerdem die \textit{Kubernetes-API}, welche als Zugang zum Cluster von außen dient, bereit und authentifiziert und autorisiert die Zugriffe.
            Kommandozeilen-Tools wie kubectl oder Drittanbieter-Software können so auf das Cluster zugreifen~\cite{kubernetes-devopscube}.
      \item[Scheduler]
            Der \textit{Scheduler} ist für die Verteilung von \textit{Containern} auf die \textit{Worker Nodes} zuständig.
            Er berücksichtigt dabei die zur Verfügung stehenden Ressourcen der einzelnen \textit{Nodes}~\cite{kubernetes-cluster-ionos}.
      \item[Controller-Manager]
            Der Controller-Manager handhabt clusterweite Funktionen zur Aufrechterhaltung des Zustands des Clusters. Er behandelt \ac{zB} den Ausfall von \textit{Nodes}
            im Cluster~\cite{kubernetes-in-action-introduction}.
      \item[etcd]
            Die \textit{etcd} ist eine konsistente Key-Value-Datenbank, die in verteilten Systemen eingesetzt werden kann~\cite{etcd-ibm}.
            Bei Kubernetes speichert \textit{etcd} die gesamte Clusterkonfiguration~\cite{kubernetes-in-action-introduction}.
\end{description}

Ein \textit{Worker Node} ist für die Ausführung von den Anwendungen in \textit{Containern} zuständig.
Es führt Aufgaben aus, die diesem von dem \textit{Master Node} zugewiesen werden~\cite{kubernetes-cluster-vmware}.
Ein \textit{Worker Node} besteht aus den folgenden zwei Komponenten:
\begin{description}
      \item[Kubelet]
            Das \textit{Kubelet} steuert die \textit{Container} auf dem \textit{Worker Node}~\cite{kubernetes-in-action-introduction}.
            Zur Ausführung der \textit{Container} wird eine \textit{Container-Runtime} genutzt.
            Das \textit{Kubelet} steht im Austausch mit dem \textit{API-Server} des \textit{Master Nodes}. Es überwacht \ac{zB} die laufenden Container
            und meldet deren Status, Ereignisse und Ressourcenverbrauch an den \textit{API-Server}~\cite{kubernetes-in-action-chapter-11}.
      \item[kube-proxy]
            Der \textit{kube-proxy} ist für die Netzwerkkonnektivität des \textit{Worker Nodes} verantwortlich.
            Er ist für die Umsetzung von Netzwerkregeln und die Lastenverteilung zuständig~\cite{kubernetes-in-action-introduction,kubernetes-cluster-vmware}.
\end{description}

\subsection{Objekte in Kubernetes}

Kubernetes Objekte sind persistente Entitäten eines Kubernetes Systems. Sie repräsentieren den aktuellen Zustand des Clusters.
Die Kubernetes Objekte können beschreiben, welche Anwendungen auf dem Cluster in \textit{Containern} laufen,
welche Ressourcen den Anwendungen zur Verfügung stehen und
wie sich die Anwendungen verhalten, \ac{zB} unter welchen Umständen eine Anwendung neu gestartet werden soll~\cite{kubernetes-objects}.
\\
Eines der Kernkonzepte von Kubernetes ist der sogenannte \textit{Desired State}~\cite{kubernetes-objects-kodekloud}.
Der Nutzer definiert mit Hilfe Kubernetes Objekte den angestrebten Zustand (\textit{Desired State}) des Clusters und Kubernetes
sorgt für die Herstellung und Aufrechterhaltung dieses Zustandes.
\\
Die Definition eines Kubernetes Objektes erfolgt in Form einer \textit{Manifest}-Datei im \ac{yaml}-Format.
In jeder \textit{Manifest}-Datei müssen folgende Attribute definiert sein~\cite{kubernetes-objects}:

\begin{itemize}
      \setlength\itemsep{-0.5cm}
      \item
            \textbf{apiVersion} Gibt an welche Version der \textit{Kubernetes API} genutzt wird, um das Objekt zu erstellen.
      \item
            \textbf{kind}  Gibt die Art des Kubernetes Objektes an.
      \item
            \textbf{metadata}  Enhält Daten die das Objekt eindeutig identifizieren, \ac{zB} ein Name und eine ID\@.
      \item
            \textbf{spec} Beschreibt den \textit{Desired State} des Objektes.
\end{itemize}

Kapitel~\ref{ch:implementation} und Kapitel~\ref{ch:comparison} nutzen spezifische Kubernetes Objekte und Konzepte die im Folgenden
zum besseren Verständnis beschrieben werden:

\begin{description}
      \item[Pod]
            Ein \textit{Pod} ist eine Gruppe von \textit{Containern}, die sich auf demselben \textit{Worker Node} befinden.
            Es wird sichergestellt, dass sich alle \textit{Container}, die in einem Pod zusammengefasst,
            immer auf dem selben \textit{Worker Node} ausgeführt werden~\cite{kubernetes-objects-kodekloud,kubernetes-in-action-chapter-3}.
      \item[Replica Set]
            Ein \textit{Replica Set} führt eine spezifizierte Anzahl von \textit{Pods} aus und sorgt dafür, dass diese Anzahl konstant bleibt.
            Sollte ein \textit{Pod} ausfallen, sorgt das \textit{Replica Set} dafür, dass ein neuer gestartet wird. Bei hoher Ressourceauslastung
            nutzt das \textit{Replica Set} eine Lastenverteilung, um alle \textit{Pods} gleichmäßig auszulasten~\cite{kubernetes-replica-set-kubematic}.
      \item[Deployment]
            Mit Hilfe eines \textit{Deployments} ist es möglich die Anwendungen in \textit{Pods}, die einem \textit{Replica Set} zugeordnet sind, zu aktualisieren.
            In der \textit{Manifest}-Datei kann dazu der \textit{Desired State} definiert werden.
            Das Ausrollen der Aktualisierung erfolgt für alle \textit{Pods} schrittweise.
            Tritt bei der Aktualisierung eines \textit{Pods} ein Fehler auf, so kann dieser neu erstellt oder wenn nötig auf den alten Stand zurückgesetzt werden.
            Kubernetes sorgt dafür, dass die Ausfallzeit der Anwendung so gering wie möglich ist~\cite{kubernetes-in-action-chapter-9,kubernetes-deployment-vmware}.
      \item[Secret]
            Bei einem \textit{Secret} handelt es sich um ein Kubernetes Objekt, das sensible Daten, wie \ac{zB} Passwörter oder SSH-Schlüssel, verschlüsselt speichert.
            Sensible Daten müssen so nicht in der \textit{Manifest}-Datei des \textit{Pods} und im Programmcode der Anwendung gespeichert werden~\cite{kubernetes-secrets,kubernetes-secrets-medium}.
      \item[Namespace]
            Ein \textit{Namespace} ermöglicht dem Nutzer ein physisches Kubernetes-Cluster in verschiedene virtuelle Cluster aufzuteilen.
            Sie können zum Einsatz kommen, wenn sich verschiedene Teams ein Kubernetes-Cluster teilen~\cite{kubernetes-namespaces-vmware}.
      \item[Label]
            Ein \textit{Label} ist ein Schlüssel-Wert-Paar, das einem Kubernetes Objekt, wie \ac{zB} einem \textit{Pod}, zugeordnet werden kann.
            \textit{Labels} können im \textit{metadata} Attribut der \textit{Manifest}-Datei eines Objektes angegeben werden.
            Der Schlüssel eines \textit{Labels} muss für ein Objekt eindeutig sein. Ein \textit{Label} kann mehreren Objekten zugeordnet werden.
            Der Nutzer hat die Möglichkeit eine Menge von Objekten über ein \textit{Label} zu selektieren~\cite{kubernetes-labels}.
\end{description}

\section{\acs{lsp}}

Das \ac{lsp} wurde 2015 von Microsoft für den Editor Visual Studio Code entwickelt.
Bei einem Editor handelt es sich in diesem Kontext um ein Computerprogramm zur Erstellung und Bearbeitung von Textdateien~\cite{editor-definition}.
Das Ziel war es, eine Schnittstelle bereitzustellen, die unter anderem die Autovervollständigung, die Code Navigation und das Syntaxhighlighting
für jeden Editor vereinheitlicht~\cite{lsp-witekio}.
Die Fähigkeit eines Editors Autovervollständigung, Code Navigation und Syntaxhighlighting zu ermöglichen wird im Folgenden als Sprachunterstützung bezeichnet.
Das \ac{lsp}-Protokoll soll vermeiden, dass eine Implementierung der Sprachunterstützung für jeden Editor separat durchgeführt werden muss.
Abbildung~\ref{fig:without-lsp} veranschaulicht dieses Problem.

\begin{figure}[htp] % htp = hier (h), top (t), oder auf einer eigenen Seite (p).
      \centering
      \includesvg[width=0.5\textwidth]{images/without-LSP.svg} % width immer angeben!
      \caption{Sprachunterstützung der Editoren ohne das \acs{lsp}}
      \label{fig:without-lsp}
\end{figure}

Als Lösung für das oben stehende Problem definiert das \ac{lsp} einen \ac{lsp}-Client und einen \ac{lsp}-Server.
Der \ac{lsp}-Client ist eine Erweiterung des Editors und sendet \ac{zB} Änderungen an Dateien zum Server.
Der \ac{lsp}-Server enthält auf der anderen Seite die Logiken zur Sprachunterstützung~\cite{lsp-witekio}.
Abbildung~\ref{fig:with-lsp} zeigt, dass mit Hilfe des \ac{lsp} die Sprachunterstützung nur einmal auf Server Seite entwickelt wird.
Für die verschiedenen Editoren muss folglich nur eine Client-Erweiterung implementiert werden, um alle Sprachen zu unterstützen.

\begin{figure}[htp] % htp = hier (h), top (t), oder auf einer eigenen Seite (p).
      \centering
      \includesvg[width=0.5\textwidth]{images/with-lsp.svg} % width immer angeben!
      \caption{Sprachunterstützung der Editoren mit \acs{lsp}}
      \label{fig:with-lsp}
\end{figure}

Die Kommunikation zwischen \ac{lsp}-Client und \ac{lsp}-Server erfolgt über Nachrichten, die jeweils aus einem \textit{Header}- und aus einem \textit{Content}-Teil bestehen.
Der \textit{Header}-Teil enthält zum einen die Länge des \textit{Content}-Teils in Byte. Zum Anderen enthält dieser eine Typenbezeichnung,
die den \ac{mime}-Typ des \textit{Content}-Teils angibt. Der \textit{Content}-Teil enthält den Inhalt der Nachricht. Die Struktur des Inhalts wird durch
das \acs{json}-\acs{rpc} Protokoll vorgegeben~\cite{lsp-microsoft,lsp-medium}.
\\
Bei \ac{rpc} handelt es sich um ein Programmierparadigma, dass die Entwicklung von verteilten Anwendungen vereinfachen soll.
Das Konzept ist dabei, dass ein Programm auf einem Client eine Methode auf einem entfernten System, einem Server, aufrufen kann,
ohne die zugrundeliegenden Netzwerkdetails zu kennen~\cite{rpc-computer-weekly}.

Das \acs{json}-\acs{rpc} Protokoll definiert für die Kommunikation drei Arten von Nachrichten~\cite{lsp-microsoft}:
\begin{description}
      \item[Request Message] Die \textit{Request Message} ist eine Nachricht vom Client an den Server.
            Sie enthält eine ID zur eindeutigen Identifikation der Nachricht und die Methode, die auf dem Server aufgerufen werden soll, sowie die Methodenparameter.
            Der Server muss eine Antwort an den Client zurücksenden.
      \item[Response Message] Die \textit{Response Message} ist die Antwort des Servers an den Client. Sie enthält die ID der dazugehörigen \textit{Request Message} und das Ergebnis der aufgerufenen Methode.
            Im Fehlerfall enthält die \textit{Response Message} kein Ergebnis, sondern eine Beschreibung des Fehlers.
      \item[Notification Message]  Die \textit{Notification Message} kann vom Client oder vom Server versendet werden.
            Der Empfänger muss auf diese Art von Nachricht nicht antworten.
            Sie enhält den Ereignis-Typ, sowie die Ereignis-Parameter.
\end{description}

Die Methoden und Ereignisse, die Client und Server zur Verfügung stellen werden als \textit{Capabilities} (Fähigkeiten) bezeichnet~\cite{lsp-medium}.
Das \ac{lsp} gibt \textit{Capabilities} vor, die jeder Client implementieren muss:
\begin{description}
      \item[textDocument/didOpen] Eine \textit{Notification Message} die dem Server signalisiert, dass ein neues Dokument geöffnet wurde.
      \item[textDocument/didChange] Eine \textit{Notification Message} die dem Server eine Änderung in einem Dokument anzeigt.
      \item[textDocument/didClose]  Eine \textit{Notification Message} die dem Server anzeigt, dass eine Dokument geschlossen wurde.
\end{description}

Beim Initialisieren der Verbindung zwischen Client und Server müssen diese jeweils der anderen Stelle ihre \textit{Capabilities} mitteilen.
Der Client sendet eine Nachricht zur Initialisierung mit seinen \textit{Capabilities} an den Server.
Dieser antwortet wiederum mit einer Nachricht, die dessen \textit{Capabilities} enthält.
Nach Erhalt dieser Nachricht sendet der Client erneut eine Nachricht an den Server um zu signalisieren, dass die Initialisierung abgeschlossen ist.
Anschließend kann der Nachrichtenaustausch zwischen Client und Server stattfinden~\cite{lsp-microsoft,lsp-medium}.

\begin{figure}[htp] % htp = hier (h), top (t), oder auf einer eigenen Seite (p).
      \centering
      \includesvg[width=0.5\textwidth]{images/lsp-sequence-diagram.svg} % width immer angeben!
      \caption{Sequenzdiagramm einer \acs{lsp}-Kommunikation}
      \label{fig:lsp-sequence-diagram}
\end{figure}

Abbildung~\ref{fig:lsp-sequence-diagram} zeigt ein Sequenzdiagramm einer Beispielkommunikation zwischen einem \ac{lsp}-Client und einem \ac{lsp}-Server.
Dabei erfolgt die Initialisierung der Verbindung mit Mitteilung der \textit{Capabilities}.
Danach teilt der Client dem Server in Form einer \textit{Notification Message} mit, dass ein Dokument geöffnet wurde.
Zuletzt sendet der Server dem Client eine \textit{Notification Message} mit dem Ergebnis einer Validierung, das \acs{zB} die Syntax Fehler in einer Datei enthalten könnte.


\section{\acs{json}-Format}\label{sec:json-format}

\ac{json} ist ein textbasiertes Format zum Austausch von Daten. Die Syntax ist von Programmiersprache Javascript abgeleitet.
Die wichtigsten Elemente mit der Verwendung von dem Beispiel auf Listing \ref{lst:json-example} erklärt \cite{json-org,json-developer-mozilla}.

\begin{listing}[htp]
      \begin{minted}[fontsize=\small]{yaml}
{
  "ray": "a drop of golden sun",
  "pi": 3.14159,
  "xmas": true,
  "french-hens": 3,
  "calling-birds": [
     "huey",
     "hello": {
       "count": 2
     }    
  ],
  "xmas-fifth-day": {
    "partridges": {
      "count": null,
    },
  }
}
      \end{minted}
      \caption{Beispiel einer \acs{json}-Datei}
      \label{lst:json-example}
\end{listing}

\begin{description}
      \setlength\itemsep{-0.5cm}
      \item[Object] TODO
      \item[Array] TODO
      \item[Key und Value] TODO
\end{description}



\section{\acs{yaml}-Format}\label{sec:yaml-format}

\ac{yaml} ist eine Sprache zur Serialisierung von Daten. Sie wird oft als Format für Konfigurationsdateien genutzt
und stellt eine Alternative zum \acs{json}-Format dar.
Listing~\ref{lst:yaml-example} zeigt eine Beispieldatei im \ac{yaml}-Format. Das Beispiel enthält die gleichen Daten, wie die
in Listing~\ref{lst:json-example} dargestellte \acs{json}-Datei.
Im Folgenden wird der Aufbau und die wichtigsten Elemente des Formats erläutert~\cite{yaml-spacelift,yaml-cloudbees,yaml-javatpoint}.

\begin{listing}[htp]
      \begin{minted}[fontsize=\small]{yaml}
---
#comment
ray: a drop of golden sun
pi: 3.14159
xmas: true
french-hens: 3
calling-birds:
  - huey
  - hello:
      count: 2
xmas-fifth-day:
  partridges:
    count: null
      \end{minted}
      \caption{Beispiel einer \acs{yaml}-Datei}
      \label{lst:yaml-example}
\end{listing}

\begin{description}
      \setlength\itemsep{-0.5cm}
      \item[Einrückung und Leerzeichen]
            Leerzeichen sind ein Teil der \acs{yaml}-Formatierung. Neue Zeilen zeigen das Ende eines Elementes an.
            Die Anzahl der Leerzeichen, welche auch als Einrückung bezeichnet wird, definiert die Hierarchie von Elementen.
            So gehört das Element in Zeile 10 \ac{zB} zu dem Element in Zeile 9.
      \item[Document]
            Eine \ac{yaml}-Datei kann mehrere sogenannte \textit{Documents} enthalten. Diese trennen die \ac{yaml}-Datei logische voneinander.
            Sie können also als mehrere voneinander unabhängige \ac{yaml}-Dateien angesehen werden.
            Eingeleitet wird ein \textit{Document}
            durch ``---'', siehe Zeile 1.
      \item[Kommentare]
            Einen einzeiliger Kommentar wird im \ac{yaml}-Format durch ``\#'' eingeleitet, siehe Zeile 2.
      \item[Mapping]
            Ein \textit{Mapping} ist eine ungeordnete Menge von Schlüssel-Werte Paaren, wobei jeder Schlüssel
            in Kontext des \textit{Mappings} einzigartig sein muss.\ \textit{Mappings} können verschachtelt werden.
            Die Verschachtelung wird mit der Einrückung angezeigt, siehe Zeile 11 bis 13.
      \item[Pair]
            Ein Schlüssel-Werte Paar eines \textit{Mappings} wird auch \textit{Pair} genannt.
      \item[Scalar]
            Ein \textit{Scalar} kann eine Zeichenkette, eine Zahl, einen boolscher Wert oder den Wert ``null'' enthalten.
            Ein \textit{Scalar} kann der Schlüssel eines \textit{Mappings} sein oder den Wert eines Elementes darstellen.
            Beispiele dafür sind in Zeile 3, 4, 5 und in Zeile 13 zu sehen.
      \item[Sequence]
            Bei einer \textit{Sequence} handelt es sich um eine ungeordnete Folge von einem oder mehreren \ac{yaml}-Elementen.
            Das Element einer \textit{Sequence} durch einen Bindestrich gefolgt von einem Leerzeichen eingeleitet.
            Es kann ein \textit{Scalar} oder ein \textit{Mapping} enthalten, siehe Zeile 8 und 9.

\end{description}


\section{\acs{json}-Schema}\label{sec:json-schema}

\acs{json}-Schema ist eine deklarative Sprache, die die Validierung des Formates und der Struktur eines \ac{json}-\textit{Object} ermöglicht.
Mit Hilfe des \acs{json}-Schemas können Regeln und Einschränkungen definiert werden, um den Aufbau eines validen \ac{json}-\textit{Objects} zu beschreiben~\cite{json-schema-description}.

\begin{listing}[htp]
      \begin{minted}[fontsize=\small]{json}
{
  "product": {
    "name": "Widget",
    "price": 10.99
  }
}
      \end{minted}
      \caption{Ein \acs{json}-Objekt}
      \label{lst:json-object}
\end{listing}

Die Regeln und Einschränkungen werden mit einem Satz von Schlüsselwörtern und Objekten im \acs{json}-Format definiert.
\\
Auf Listing~\ref{lst:json-object} ist ein Beispiel eines \acs{json}-\textit{Objects} abgebildet, dass mit Hilfe des auf Listing~\ref{lst:json-schema}
dargestellten \acs{json}-Schemas validiert werden kann.
\\
Es folgt eine Erklärung der auf Listing ~\ref{lst:json-schema} verwendeten Schlüsselwörter~\cite{json-schema-creation}:

\begin{itemize}
      \setlength\itemsep{-0.5cm}
      \item
            \textbf{\$schema} Gibt die Version der \acs{json}-Schema-Spezifikation. Unterschiedliche Versionen der Spezifikation
            unterstützen andere Schlüsselwörter.
      \item
            \textbf{type} Legt den Typ des Attributes im \acs{json}-Format fest.
      \item
            \textbf{properties} Gibt an, welche Attribute ein \acs{json}-Objekt enthalten kann.
      \item
            \textbf{required} Legt fest, welche Attribute ein \acs{json}-Objekt enthalten muss.
      \item
            \textbf{minimum} Gibt den kleinsten Wert an, den ein numerischer Wert enthalten darf.
\end{itemize}

Anhand der im \acs{json}-Schema enthaltenen Informationen kann ein Algorithmus die Validierung des Objektes durchführen.
Die \acs{json}-Schema-Spezifikation enthält noch weitere Schlüsselwörter, die aufgrund des Umfangs nicht erläutert werden.
\newpage
\begin{listing}[htp]
      \begin{minted}[fontsize=\small]{json}
{
  "$schema": "https://json-schema.org/draft/2020-12/schema",
  "title": "Product",
  "type": "object",
  "properties": {
      "product": {
        "type": "object",
        "properties": {
          "name": {
            "type": "string"
          },
          "price": {
            "type": "number", 
            "minimum": 0
          },
        },
        "required": ["name", "price"]
      }
  },
  "required": ["product"]
}
      \end{minted}
      \caption{Ein \acs{json}-Schema}
      \label{lst:json-schema}
\end{listing}

\section{Parser}\label{sec:parser}

Bei einem Parser handelt es sich um ein Computerprogramm, das Eingaben, bei denen es sich in der Regel um Code handelt,
zerlegen und in ein anderes Format umwandeln kann, damit
diese weiterverarbeitet werden können. Der Parser geht dabei in zwei Schritten vor~\cite{parser-tech-target}:

\begin{description}
      \item[1. Lexikalische Analyse]
            Die Eingabe wird in kleinere Teile zerlegt. Dabei wird diese in Zeichenfolgen, sogenannte \text{Token}
            gruppiert, die der Parser als nicht weiter teilbar kennt. Bei dem \ac{json}-Format würde dies \ac{zB} einer geschweiften Klammer oder
            dem Namen eines Attributes entsprechen.
      \item[2. Syntaktische Analyse]
            Die bei der lexikalischen Analyse erstellten \text{Token} werden in diesem Schritt in einer Baumstruktur gespeichert, sodass die
            Baumstruktur auch die Eingabereihenfolge der \text{Token} berücksichtigt. Diese Baumstruktur wird \ac{ast} genannt.
            Anhand des \ac{ast} überprüft der Parser, ob die Eingabe Syntaxfehler enthält.
\end{description}
