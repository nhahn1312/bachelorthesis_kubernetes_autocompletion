\chapter{Technische Grundlagen}\label{ch:technical-basics}

\section{\acs{lsp}}

Das \ac{lsp} wurde 2015 von Microsoft für den Editor Visual Studio Code entwickelt.
Bei einem Editor handelt es sich in diesem Kontext um ein Computerprogramm zur Erstellung und Bearbeitung von Textdateien~\cite{editor-definition}.
Das Ziel war es, eine Schnittstelle bereitzustellen, die unter anderem die Autovervollständigung, die Code Navigation und das Syntaxhighlighting
für jeden Editor vereinheitlicht~\cite{lsp-witekio}.
Die Fähigkeit eines Editors Autovervollständigung, Code Navigation und Syntaxhighlighting zu ermöglichen wird im Folgenden als Sprachunterstützung bezeichnet.
Das \ac{lsp}-Protokoll soll vermeiden, dass eine Implementierung der Sprachunterstützung für jeden Editor seperat durchgeführt werden muss.
Abbildung \ref{fig:without-lsp} veranschaulicht dieses Problem.

\begin{figure}[htp] % htp = hier (h), top (t), oder auf einer eigenen Seite (p).
    \centering
    \includesvg[width=0.5\textwidth]{images/without-LSP.svg} % width immer angeben!
    \caption{Sprachunterstützung der Editoren ohne das \acs{lsp}}
    \label{fig:without-lsp}
\end{figure}

Als Lösung für das oben stehende Problem definiert das \ac{lsp} einen \ac{lsp}-Client und einen \ac{lsp}-Server.
Der \ac{lsp}-Client ist eine Erweiterung des Editors und sendet \ac{zB} Änderungen an Dateien zum Server.
Der \ac{lsp}-Server enthält auf der anderen Seite die Logiken zur Sprachunterstützung~\cite{lsp-witekio}.
Abbildung~\ref{fig:with-lsp} zeigt, dass mit Hilfe des \ac{lsp} die Sprachunterstützung nur einmal auf Server Seite entwickelt wird.
Für die verschiedenen Editoren muss folglich nur eine Client Erweiterung implementiert werden, um alle Sprachen zu unterstützen.

\begin{figure}[htp] % htp = hier (h), top (t), oder auf einer eigenen Seite (p).
    \centering
    \includesvg[width=0.5\textwidth]{images/with-lsp.svg} % width immer angeben!
    \caption{Sprachunterstützung der Editoren mit \acs{lsp}}
    \label{fig:with-lsp}
\end{figure}

Das Kommunikation zwischen \ac{lsp}-Client und \ac{lsp}-Server erfolgt über Nachrichten, die jeweils aus einem \textit{Header}- und aus einem \textit{Content}-Teil bestehen.
Der \textit{Header}-Teil enthält zum einen die Länge des \textit{Content}-Teils in Byte. Zum Anderen enthält dieser eine Typenbezeichnung,
die den \ac{mime}-Typ des \textit{Content}-Teils angibt. Der \textit{Content}-Teil enhält den Inhalt der Nachricht. Die Struktur des Inhalts wird durch
das \acs{json}-\acs{rpc} Protokoll vorgegeben~\cite{lsp-microsoft,lsp-medium}.

Das \acs{json}-\acs{rpc} Protokoll definiert für die Kommunikation drei Arten von Nachrichten:
\begin{description}
    \item[Request Message] Die \textit{Request Message} ist eine Nachricht vom Client an den Server.
          Sie enthält eine ID zur eindeutigen Identifikation der Nachricht und die Methode, die auf dem Server aufgerufen werden soll, sowie die Methodenparameter.
          Der Server muss eine Antwort an den Client zurücksenden~\cite{lsp-microsoft}.
    \item[Response Message] Die \textit{Response Message} ist die Antwort des Servers an den Client. Sie enthält die ID der dazugehörigen \textit{Request Message} und das Ergebnis der aufgerufenen Methode.
          Im Fehlerfall enthält die \textit{Response Message} kein Ergebnis, sondern eine Beschreibung des Fehlers~\cite{lsp-microsoft}.
    \item[Notification Message]  Die \textit{Notification Message} kann vom Client oder vom Server versendet werden.
          Der Empfänger muss auf diese Art von Nachricht nicht antworten.
          Sie enhält den Ereignis-Typ, sowie die Ereignis-Parameter. Der Server muss keine \textit{Response Message} zurücksenden~\cite{lsp-microsoft}.
\end{description}

Die Methoden und Ereignisse die Client und Server zur Verfügung stellen werden als \textit{Capabilities} (Fähigkeiten) bezeichnet~\cite{lsp-medium}.
Das \ac{lsp} gibt \textit{Capabilities} vor, die jeder Client implementieren muss:
\begin{description}
    \item[textDocument/didOpen] Eine \textit{Notification Message} die dem Server signalisiert, dass ein neues Dokument geöffnet wurde.
    \item[textDocument/didChange] Eine \textit{Notification Message} die dem Server eine Änderung in einem Dokument anzeigt.
    \item[textDocument/didClose]  Eine \textit{Notification Message} die dem Server anzeigt, dass eine Dokument geschlossen wurde.
\end{description}

Beim Initialisieren der Verbindung zwischen Client und Server müssen diese jeweils der anderen Stelle ihre \textit{Capabilities} mitteilen.
Der Client sendet eine Nachricht zur Intialisierung mit seinen \textit{Capabilities} an den Server.
Dieser antwortet wiederum mit einer Nachricht, die dessen \textit{Capabilities} enthält.
Nach Erhalt dieser Nachricht sendet der Client erneut eine Nachricht an den Server um zu signalisieren, dass die Intialisierung abgeschlossen ist.
Anschließend kann Nachrichtenaustausch zwischen Client und Server beginnen\cite{lsp-microsoft,lsp-medium}.

\begin{figure}[htp] % htp = hier (h), top (t), oder auf einer eigenen Seite (p).
    \centering
    \includesvg[width=0.5\textwidth]{images/lsp-sequence-diagram.svg} % width immer angeben!
    \caption{Sequenzdiagramm einer \acs{lsp}-Kommunikation}
    \label{fig:lsp-sequence-diagram}
\end{figure}

Abbildung~\ref{fig:lsp-sequence-diagram} zeigt ein Sequenzdiagramm einer Beispielkommunikation zwischen einem \ac{lsp}-Client und einem \ac{lsp}-Server.
Dabei erfolgt die Intialisierung der Kommunikation. Danach teilt der Client dem Server in Form einer \textit{Notification Message} mit, dass ein Dokument geöffnet wurde.
Zuletzt sendet der Server dem Client eine \textit{Notification Message} mit dem Ergebnis einer Validierung, das \acs{zB} die Syntax Fehler in einer Datei enhalten könnte.

\section{Kubernetes}

\section{\acs{json}-Schema}