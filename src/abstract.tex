\chapter*{Zusammenfassung}

% Inhaltsverzeichnis und Kopfzeile
\addcontentsline{toc}{chapter}{Zusammenfassung}
\markboth{Zusammenfassung}{Zusammenfassung}

Diese Bachelorarbeit befasst sich mit der Erstellung von Konfigurationsdateien für die Open-Source-Software Kubernetes.
Dabei werden Probleme und Herausforderungen bei der Erstellung beleuchtet und bereits bestehende Lösungen
für die Probleme identifiziert. Dafür wird eine Stichwortsuche genutzt.
Im Rahmen der Suche wurde festgestellt, dass keine wissenschaftliche Arbeit existiert, die sich empirisch mit den Problemen und Herausforderungen
beim Erstellen der Konfigurationsdateien auseinandergesetzt.
Daher wurde eine Websuche durchgeführt, um diese zu erkennen.
Mit Hilfe eines Laborversuchs wurden Probleme erfasst, die von den Anwendungen nicht gelöst werden.
Hier konnte gezeigt werden, dass keine der existierenden Lösungen alle Probleme lösen kann.
Die Autovervollständigung und Validierung von Ressourcen auf dem Cluster in den Konfigurationsdateien wird von den Lösungen nicht angeboten.
Es wurde anschließend eine Anwendung implementiert, die diese Funktionalität für auswählte Cluster-Ressourcen bietet.
Die implementierte Anwendung konnte die funktionalen Anforderungen erfüllen.
Abschließend wurde diese mit den bestehenden Lösungen auf Basis von Kriterien zur Benutzerfreundlichkeit für statische Code-Analysetools verglichen.
Der Vergleich hat gezeigt, dass keine der untersuchten Anwendungen mehr als $50\%$ der Kriterien erfüllen konnte.
Die eigene Anwendung erfüllt nur 5 von 26 gestellten Kriterien.

