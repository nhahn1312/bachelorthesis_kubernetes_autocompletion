% Select input encodung, usually utf8 is the best choice, on windows, \usepackage[latin1]{inputenc} maybe required
\usepackage[utf8]{inputenc}
\usepackage[T1]{fontenc}
\usepackage[ngerman]{babel}
\usepackage{csquotes}
\usepackage{xcolor}
\usepackage[inkscapeformat=png]{svg}

\MakeOuterQuote{"} % Damit ist es möglich, " " zu verwenden ohne Umlaut zu erzeugen
\defaulthyphenchar=127 % Dadurch werden auch Wörter mit Bindestrich getrennt, die schon Bindestriche enthalten.

% geometry
\usepackage[bindingoffset=1cm, left=2.5cm, right=2.5cm, top=2.5cm, bottom=2.5cm]{geometry}

% Headline
\usepackage{fancyhdr}
\pagestyle{fancy}
\renewcommand{\chaptermark}[1]{\markboth{\thechapter\ #1}{}}
\lhead{\leftmark} \rhead{\thepage}
\cfoot{}
\fancypagestyle{plain}{}

\RedeclareSectionCommand[beforeskip=1.5cm,afterskip=1cm]{chapter}

\usepackage[capposition=bottom]{floatrow}
\usepackage{float}

%% Listing usd for code in table
\usepackage{listings}

\definecolor{yamlkeygreen}{RGB}{32,128,18}

\newcommand\YAMLcolonstyle{\color{black}\mdseries}
\newcommand\YAMLkeystyle{\color{yamlkeygreen}\bfseries}
\newcommand\YAMLvaluestyle{\color{black}\mdseries}

\makeatletter

% here is a macro expanding to the name of the language
% (handy if you decide to change it further down the road)
\newcommand\language@yaml{yaml}

\expandafter\expandafter\expandafter\lstdefinelanguage
\expandafter{\language@yaml}
{
keywords={true,false,null,y,n},
keywordstyle=\color{darkgray}\bfseries,
basicstyle=\footnotesize\YAMLkeystyle,                                 % assuming a key comes first
sensitive=false,
comment=[l]{\#},
morecomment=[s]{/*}{*/},
commentstyle=\color{purple}\ttfamily,
stringstyle=\YAMLvaluestyle\ttfamily,
moredelim=[l][\color{orange}]{\&},
moredelim=[l][\color{magenta}]{*},
moredelim=**[il][\YAMLcolonstyle{:}\YAMLvaluestyle]{:},   % switch to value style at :
moredelim=**[is][\color{yamlkeygreen}]{<}{>}
morestring=[b]',
morestring=[b]",
literate =    {---}{{\ProcessThreeDashes}}3
%{>}{{\textcolor{red}\textgreater}}1
{|}{{\color{black}\mdseries|}}1
{\ -\ }{{\mdseries\ -\ }}3,
}

% switch to key style at EOL
\lst@AddToHook{EveryLine}{\ifx\lst@language\language@yaml\YAMLkeystyle\fi}
\makeatother

\newcommand\ProcessThreeDashes{\llap{\color{blue}\mdseries-{-}-}}

% Colors
\usepackage{color}
\usepackage{colortbl}

% Tables
\usepackage{tabularx}
\usepackage{multirow}
\usepackage{makecell}
\setlength{\tabcolsep}{4pt}

% Drawing graphs etc.
\usepackage{pgf}
\usepackage{tikz}
\usetikzlibrary{arrows,automata}

%add command for circle drawing in prepartion analysis table
\newcommand*\emptycirc[1][1ex]{\tikz\draw (0,0) circle (#1);}
\newcommand*\halfcirc[1][1ex]{%
  \begin{tikzpicture}
    \draw[fill] (0,0)-- (90:#1) arc (90:270:#1) -- cycle ;
    \draw (0,0) circle (#1);
  \end{tikzpicture}}
\newcommand*\fullcirc[1][1ex]{\tikz\fill (0,0) circle (#1);}

% Footnotes
\usepackage{footmisc}

\usepackage{xspace}
\newcommand{\sic}{[\acs{sic}]\xspace}

% math
\usepackage{amsmath}
\usepackage{siunitx}

% lists
\usepackage{paralist}
\usepackage{enumitem}

% Figures
\usepackage{graphicx, wrapfig}

% Hyperlinks
\usepackage[hyphens]{url}
\usepackage{hyperref}
\hypersetup{colorlinks, citecolor=black, linkcolor=black, urlcolor=black}

% Minted
\usepackage[chapter]{minted}
%\usemintedstyle{xcode}
\setminted{frame=single,tabsize=2,linenos,autogobble}

\newmintinline[code]{text}{breaklines}

\newminted[mdcodeblock]{md}{autogobble,frame=none,linenos=false,breaklines}
\usemintedstyle[js]{abap}
\renewcommand\theFancyVerbLine{\normalsize\arabic{FancyVerbLine}}

\makeatletter
\def\myinline#1#2{%
  \ifx\@footnotetext\TX@trial@ftn
    \detokenize{#2}%
  \else
    \mintinline{#1}{#2}%
  \fi}
\makeatother

% list of abbreviations
\usepackage[printonlyused]{acronym}

% Set line pitch
\usepackage{setspace}
\onehalfspacing              % anderthalbzeilig (oder auch \doublespace)

%fancyBox
%\usepackage{fancybox}
\usepackage{placeins}

% Layout corrections (Schusterjungen)
\clubpenalty = 10000
% Layout corrections (Hurenkinder)
\widowpenalty = 10000
\displaywidowpenalty = 10000

% Figures
\usepackage{caption}
\usepackage[hypcap=true,labelformat=simple]{subcaption}
\renewcommand{\thesubfigure}{(\alph{subfigure})}

% Tables
\usepackage{booktabs}

% Frequently used column types
\newcolumntype{C}[1]{>{\centering\arraybackslash}p{#1}} % centering column type with fixed width
\newcolumntype{R}[1]{>{\raggedleft\arraybackslash}p{#1}} % right aligned column type with fixed width
\newcolumntype{L}[1]{>{\raggedright\arraybackslash}p{#1}} % left aligned column type with fixed width

% Shortcuts for referencing floats:
\newcommand{\fig}[1]{\figurename~\ref{#1}} %shortcut for a figure reference
\newcommand{\tab}[1]{Table~\ref{#1}} %shortcut for a table reference
\newcommand{\eq}[1]{(\ref{#1})} %shortcut for an equation reference
\newcommand{\lst}[1]{Listing~\ref{#1}} %shortcut for a listing reference
\newcommand{\sect}[1]{Section~\ref{#1}} %shortcut for a Section reference

\counterwithout{figure}{chapter}
\counterwithout{listing}{chapter}
\counterwithout{table}{chapter}